%% Generated by Sphinx.
\def\sphinxdocclass{report}
\documentclass[letterpaper,10pt,english]{sphinxmanual}
\ifdefined\pdfpxdimen
   \let\sphinxpxdimen\pdfpxdimen\else\newdimen\sphinxpxdimen
\fi \sphinxpxdimen=.75bp\relax

\PassOptionsToPackage{warn}{textcomp}
\usepackage[utf8]{inputenc}
\ifdefined\DeclareUnicodeCharacter
% support both utf8 and utf8x syntaxes
  \ifdefined\DeclareUnicodeCharacterAsOptional
    \def\sphinxDUC#1{\DeclareUnicodeCharacter{"#1}}
  \else
    \let\sphinxDUC\DeclareUnicodeCharacter
  \fi
  \sphinxDUC{00A0}{\nobreakspace}
  \sphinxDUC{2500}{\sphinxunichar{2500}}
  \sphinxDUC{2502}{\sphinxunichar{2502}}
  \sphinxDUC{2514}{\sphinxunichar{2514}}
  \sphinxDUC{251C}{\sphinxunichar{251C}}
  \sphinxDUC{2572}{\textbackslash}
\fi
\usepackage{cmap}
\usepackage[T1]{fontenc}
\usepackage{amsmath,amssymb,amstext}
\usepackage{babel}



\usepackage{times}
\expandafter\ifx\csname T@LGR\endcsname\relax
\else
% LGR was declared as font encoding
  \substitutefont{LGR}{\rmdefault}{cmr}
  \substitutefont{LGR}{\sfdefault}{cmss}
  \substitutefont{LGR}{\ttdefault}{cmtt}
\fi
\expandafter\ifx\csname T@X2\endcsname\relax
  \expandafter\ifx\csname T@T2A\endcsname\relax
  \else
  % T2A was declared as font encoding
    \substitutefont{T2A}{\rmdefault}{cmr}
    \substitutefont{T2A}{\sfdefault}{cmss}
    \substitutefont{T2A}{\ttdefault}{cmtt}
  \fi
\else
% X2 was declared as font encoding
  \substitutefont{X2}{\rmdefault}{cmr}
  \substitutefont{X2}{\sfdefault}{cmss}
  \substitutefont{X2}{\ttdefault}{cmtt}
\fi


\usepackage[Bjarne]{fncychap}
\usepackage{sphinx}

\fvset{fontsize=\small}
\usepackage{geometry}


% Include hyperref last.
\usepackage{hyperref}
% Fix anchor placement for figures with captions.
\usepackage{hypcap}% it must be loaded after hyperref.
% Set up styles of URL: it should be placed after hyperref.
\urlstyle{same}


\usepackage{sphinxmessages}
\setcounter{tocdepth}{4}
\setcounter{secnumdepth}{4}


\title{Beamon SAD}
\date{Jan 14, 2021}
\release{alpha}
\author{Maher Albezem}
\newcommand{\sphinxlogo}{\vbox{}}
\renewcommand{\releasename}{Release}
\makeindex
\begin{document}

\pagestyle{empty}
\sphinxmaketitle
\pagestyle{plain}
\sphinxtableofcontents
\pagestyle{normal}
\phantomsection\label{\detokenize{index::doc}}


Beamon is an open\sphinxhyphen{}source based development in the institute of structural analysis and
lightweight design at RWTH University in Aachen. Beamon Static Analysis Design (SAD) is mainly
a graphical finite element program for modeling static structural systems and simulating those using
the direct stiffness method.


\chapter{Beamon In A Glance}
\label{\detokenize{index:beamon-in-a-glance}}
The following Use\sphinxhyphen{}Case\sphinxhyphen{}Diagram summarizes Beamon’s capabilities.

\noindent\sphinxincludegraphics{{beamon_capabilities}.png}

Users or specifically designers are capable of defining node coordinates and structure boundary conditions (Structural support).

Elements are connection information between nodes. In this context, a static model should build a
\sphinxhref{https://en.wikipedia.org/wiki/Graph\_(discrete\_mathematics)}{Graph}.

Each element has modifiable local system orientation and properties, profiles.

Designers can apply external forces on structures such as, nodal forces or constant element forces.

Geometries can be exported and imported as modifiable text files
(\sphinxhref{https://de.wikipedia.org/wiki/American\_Standard\_Code\_for\_Information\_Interchange}{ASCII}\sphinxhyphen{}Symbols).

Simulation results summed up in section forces and displacements for each element, and support forces and displacements
for each node can be viewed in tables and exported as Excel or CSV files.


\chapter{Documentations Contents}
\label{\detokenize{index:documentations-contents}}

\section{Getting Started}
\label{\detokenize{getting_started:getting-started}}\label{\detokenize{getting_started::doc}}

\subsection{Installation}
\label{\detokenize{getting_started:installation}}

\subsubsection{Windows Executable}
\label{\detokenize{getting_started:windows-executable}}
Download windows executable in \sphinxstylestrong{releases} directory an run it.


\subsubsection{Python Package (for developers and testers)}
\label{\detokenize{getting_started:python-package-for-developers-and-testers}}
Use the package manager \sphinxtitleref{pip\textless{}https://pip.pypa.io/en/stable/\textgreater{}} to install requirements from requirements.txt file.


\subsection{Starting Beamon}
\label{\detokenize{getting_started:starting-beamon}}

\subsubsection{How to use beamon package}
\label{\detokenize{getting_started:how-to-use-beamon-package}}
To run Beamon package use the following command in your python environment:

\begin{sphinxVerbatim}[commandchars=\\\{\}]
\PYG{n}{python} \PYG{o}{\PYGZhy{}}\PYG{n}{m} \PYG{n}{beamon}
\end{sphinxVerbatim}

This will execute the user interface of Beamon


\subsubsection{Beamon’s comman line arguments}
\label{\detokenize{getting_started:beamon-s-comman-line-arguments}}
Available command line arguments are:
\begin{enumerate}
\sphinxsetlistlabels{\arabic}{enumi}{enumii}{}{.}%
\item {} 
\sphinxhyphen{}i \textless{}path to geometry file\textgreater{}

\item {} 
\sphinxhyphen{}o \textless{}path to output geometry file\textgreater{} \sphinxstylestrong{(will be removed in future releases)}

\item {} 
\sphinxhyphen{}t test mode (lunches all submodules on start)

\item {} 
\sphinxhyphen{}r ram mode (runs database on RAM)

\item {} 
\sphinxhyphen{}d \textless{}path to database file\textgreater{}

\end{enumerate}


\subsection{Using Beamon}
\label{\detokenize{getting_started:using-beamon}}

\subsubsection{Loading a Project and Geometry files}
\label{\detokenize{getting_started:loading-a-project-and-geometry-files}}
Upon starting Beamon, the user will be confronted with a wizard to load an existing project or make a new one.
The user will also have a choice to load a geometry file directly. The user can append the geometry file to existing
data in project or overwrite all existing geometry data.

The wizard is simplified in the following diagram:

\noindent\sphinxincludegraphics{{loader_wizard}.jpg}

After loading the project (with a geometry file) a notification will show up and notify the user if
everything went well or if problems occurred.


\section{Making A Model}
\label{\detokenize{making_a_model:making-a-model}}\label{\detokenize{making_a_model::doc}}
There are two ways to define Geometry information. Either with an input file (geometry file) or using data
tables in the user interface.


\subsection{Nodes And Boundry Conditions}
\label{\detokenize{making_a_model:nodes-and-boundry-conditions}}\label{\detokenize{making_a_model:nodes-input}}
Nodes are labeled with index numbers and have coordinates in Room.
Each node has 6 degrees of freedom (DOF) defined in order \(u_x, u_y, u_z, \varphi_x, \varphi_y, \varphi_z\)
You can set boundary conditions by locking movement in the desired direction.


\subsubsection{With Geometry File}
\label{\detokenize{making_a_model:with-geometry-file}}
Nodes entry begins with keyword ‘{\color{red}\bfseries{}*}node’.
To lock a movement in a certain direction use 1 and to keep the movement unlocked use 0.
The following syntax diagram shows the syntax for nodes input.

\begin{figure}[htbp]
\centering
\capstart

\noindent\sphinxincludegraphics{{nodeEntry}.png}
\caption{Syntax diagram for nodes entry in geometry file}\label{\detokenize{making_a_model:id17}}\end{figure}

Where dof is defined as:

\begin{figure}[htbp]
\centering
\capstart

\noindent\sphinxincludegraphics{{dof}.png}
\caption{Syntax diagram for dof in nodes entry in geometry file}\label{\detokenize{making_a_model:id18}}\end{figure}

\sphinxstylestrong{Example}:

\begin{sphinxVerbatim}[commandchars=\\\{\}]
\PYG{o}{*}\PYG{n}{node}
\PYG{o}{\PYGZhy{}}\PYG{l+m+mf}{1.0}\PYG{p}{,}\PYG{l+m+mf}{0.0}\PYG{p}{,}\PYG{l+m+mf}{0.0}\PYG{p}{,}\PYG{l+m+mi}{1}\PYG{p}{,}\PYG{l+m+mi}{1}\PYG{p}{,}\PYG{l+m+mi}{1}\PYG{p}{,}\PYG{l+m+mi}{1}\PYG{p}{,}\PYG{l+m+mi}{1}\PYG{p}{,}\PYG{l+m+mi}{1}
\PYG{l+m+mf}{0.0}\PYG{p}{,}\PYG{l+m+mf}{0.0}\PYG{p}{,}\PYG{o}{\PYGZhy{}}\PYG{l+m+mf}{1.0}\PYG{p}{,}\PYG{l+m+mi}{1}\PYG{p}{,}\PYG{l+m+mi}{1}\PYG{p}{,}\PYG{l+m+mi}{1}\PYG{p}{,}\PYG{l+m+mi}{0}\PYG{p}{,}\PYG{l+m+mi}{0}\PYG{p}{,}\PYG{l+m+mi}{0}
\PYG{l+m+mf}{1.0}\PYG{p}{,}\PYG{l+m+mf}{0.0}\PYG{p}{,}\PYG{l+m+mf}{2.0}
\PYG{l+m+mf}{3.0}\PYG{p}{,}\PYG{l+m+mf}{0.0}\PYG{p}{,}\PYG{o}{\PYGZhy{}}\PYG{l+m+mf}{2.0}\PYG{p}{,}\PYG{l+m+mi}{1}\PYG{p}{,}\PYG{l+m+mi}{1}\PYG{p}{,}\PYG{l+m+mi}{1}\PYG{p}{,}\PYG{l+m+mi}{0}\PYG{p}{,}\PYG{l+m+mi}{1}\PYG{p}{,}\PYG{l+m+mi}{1}
\PYG{l+m+mf}{4.0}\PYG{p}{,}\PYG{l+m+mf}{0.0}\PYG{p}{,}\PYG{l+m+mf}{1.0}\PYG{p}{,}\PYG{l+m+mi}{1}\PYG{p}{,}\PYG{l+m+mi}{1}\PYG{p}{,}\PYG{l+m+mi}{1}\PYG{p}{,}\PYG{l+m+mi}{1}\PYG{p}{,}\PYG{l+m+mi}{0}\PYG{p}{,}\PYG{l+m+mi}{1}
\PYG{l+m+mf}{2.0}\PYG{p}{,}\PYG{l+m+mf}{0.0}\PYG{p}{,}\PYG{l+m+mf}{2.0}\PYG{p}{,}\PYG{l+m+mi}{1}\PYG{p}{,}\PYG{l+m+mi}{1}\PYG{p}{,}\PYG{l+m+mi}{1}\PYG{p}{,}\PYG{l+m+mi}{1}\PYG{p}{,}\PYG{l+m+mi}{1}\PYG{p}{,}\PYG{l+m+mi}{0}
\PYG{l+m+mf}{3.0}\PYG{p}{,}\PYG{l+m+mf}{0.0}\PYG{p}{,}\PYG{l+m+mf}{2.0}\PYG{p}{,}\PYG{l+m+mi}{1}\PYG{p}{,}\PYG{l+m+mi}{1}\PYG{p}{,}\PYG{l+m+mi}{1}\PYG{p}{,}\PYG{l+m+mi}{1}\PYG{p}{,}\PYG{l+m+mi}{1}\PYG{p}{,}\PYG{l+m+mi}{1}
\end{sphinxVerbatim}

first node location is (\sphinxhyphen{}1,0,0) and has it’s all dof directions locked. Second node location is (0,0,\sphinxhyphen{}1) and has all
translation dof directions locked.

Notice that the third node has only its coordinates defined. All undefined dof numbers will be substituted
with 0 (unlocked).


\subsubsection{With User Interface}
\label{\detokenize{making_a_model:with-user-interface}}
As in nodes, elements and other input tables
right click on a table row or on an emtpy table to add/remove rows.

Use the checkbox to lock (checked) and
unlock (unchecked) the movement in a certain direction.

\noindent\sphinxincludegraphics{{NodeEntryGUI}.png}


\subsubsection{Nodes and Boundry Conditions Visualization}
\label{\detokenize{making_a_model:nodes-and-boundry-conditions-visualization}}
Nodes are black square points. Orange cones are the boundary conditions (bc) for locking translation in each global direction
in the cartesian coordinate system. Two nested blue cones are bc for locking rotation in each direction.

\noindent\sphinxincludegraphics{{vis_node}.png}


\subsection{Elements}
\label{\detokenize{making_a_model:elements}}
Elements are, in this current development stage, 3D Beams. Each element has an index number and is defined using starting
and ending node index number. Vector \(v\) is used to define local z\sphinxhyphen{}axis. (for further info about element
orientation see {\hyperref[\detokenize{theory::doc}]{\sphinxcrossref{\DUrole{doc}{Theory}}}})


\subsubsection{With Geometry File}
\label{\detokenize{making_a_model:id3}}
Elements entry begins with keyword ‘{\color{red}\bfseries{}*}ßelement’. A starting and ending node index must be given.
Element orientation is optional input and is defined using vector \(\vec{v}\).
If not given default element orientation will be used.
At last optional but necessary profile number should be given.

The following syntax diagram shows the syntax for elements input.

\begin{figure}[htbp]
\centering
\capstart

\noindent\sphinxincludegraphics{{elementEntry}.png}
\caption{Syntax diagram for element entry in geometry file}\label{\detokenize{making_a_model:id19}}\end{figure}

\sphinxstylestrong{Example}:
from nodes example above we can define following elements:

\begin{sphinxVerbatim}[commandchars=\\\{\}]
\PYG{o}{*}\PYG{n}{element}
\PYG{l+m+mi}{1}\PYG{p}{,}\PYG{l+m+mi}{2}\PYG{p}{,}\PYG{l+m+mi}{1}\PYG{p}{,}\PYG{o}{\PYGZhy{}}\PYG{l+m+mi}{1}\PYG{p}{,}\PYG{l+m+mi}{0}\PYG{p}{,}\PYG{l+m+mi}{0}
\PYG{l+m+mi}{2}\PYG{p}{,}\PYG{l+m+mi}{3}\PYG{p}{,}\PYG{p}{,}\PYG{l+m+mi}{0}\PYG{p}{,}\PYG{l+m+mi}{0}\PYG{p}{,}\PYG{l+m+mi}{1}
\PYG{l+m+mi}{3}\PYG{p}{,}\PYG{l+m+mi}{4}\PYG{p}{,}\PYG{l+m+mi}{1}\PYG{p}{,}\PYG{o}{\PYGZhy{}}\PYG{l+m+mi}{1}\PYG{p}{,}\PYG{l+m+mi}{0}\PYG{p}{,}\PYG{l+m+mi}{0}
\end{sphinxVerbatim}

first element is defined with nodes 1 and 2 and has profile 1 and orientation vector (1,\sphinxhyphen{}1,0).
Second element has no profile defined and will be excluded from simulation later.


\subsubsection{With User Interface}
\label{\detokenize{making_a_model:id6}}
Analogous to {\hyperref[\detokenize{making_a_model:nodes-input}]{\sphinxcrossref{\DUrole{std,std-ref}{Nodes And Boundry Conditions}}}}


\subsubsection{Elements Visualization}
\label{\detokenize{making_a_model:elements-visualization}}
Visual example of an element defined with nodes 5 and 6. With local axes orientation different from global axes.

\noindent\sphinxincludegraphics{{vis_element}.png}


\subsection{Nodes Loads}
\label{\detokenize{making_a_model:nodes-loads}}
Nodes Loads or Point Load describes the force that acts at a point.
Point loads can be only defined using an existing node index number and the following parameters:

\(\vec{F}= \left( \begin{array}{c} F_x \\ F_y \\ F_z \end{array}\right)\) is Force vector
and
\(\vec{M}= \left( \begin{array}{c} M_x \\ M_y \\ M_z \end{array}\right)\) is momentum vector on the specified node


\subsubsection{With Geometry File}
\label{\detokenize{making_a_model:id7}}
To define node loads use the keyword ‘{\color{red}\bfseries{}*}load’. Each entry starts with a valid node index number followed by 6
floating\sphinxhyphen{}point numbers for force and momentum vector respectively.

The following syntax diagram hopefully clarifies syntax for point load.

\begin{figure}[htbp]
\centering
\capstart

\noindent\sphinxincludegraphics{{loadEntry}.png}
\caption{Syntax diagram for load entry in geometry files}\label{\detokenize{making_a_model:id20}}\end{figure}

\sphinxstylestrong{Example}:
from nodes example above we can define following loads:

\begin{sphinxVerbatim}[commandchars=\\\{\}]
\PYG{o}{*}\PYG{n}{load}
\PYG{l+m+mi}{2}\PYG{p}{,}\PYG{l+m+mi}{10000}\PYG{p}{,}\PYG{l+m+mi}{0}\PYG{p}{,}\PYG{l+m+mi}{0}\PYG{p}{,}\PYG{l+m+mi}{0}\PYG{p}{,}\PYG{l+m+mi}{0}\PYG{p}{,}\PYG{l+m+mi}{0}
\PYG{l+m+mi}{2}\PYG{p}{,} \PYG{l+m+mi}{1}\PYG{p}{,}\PYG{l+m+mi}{1}
\PYG{l+m+mi}{1}\PYG{p}{,} \PYG{l+m+mi}{1}\PYG{p}{,}\PYG{l+m+mi}{1}\PYG{p}{,}\PYG{l+m+mi}{1}\PYG{p}{,}\PYG{l+m+mi}{1}\PYG{p}{,}\PYG{l+m+mi}{1}\PYG{p}{,}\PYG{l+m+mi}{1}
\end{sphinxVerbatim}

Note that loads on node 1 are superposition in the calculation.
The second entry contains only the first two components of the force vector.
In this case, all other components will be assumed to be zero as if ‘2,1,1,0,0,0,0’ where given.


\subsubsection{With User Interface}
\label{\detokenize{making_a_model:id10}}
Analogous to {\hyperref[\detokenize{making_a_model:nodes-input}]{\sphinxcrossref{\DUrole{std,std-ref}{Nodes And Boundry Conditions}}}}


\subsubsection{Nodes Loads Visualization}
\label{\detokenize{making_a_model:nodes-loads-visualization}}
for \(\vec{F}= \left(\begin{array}{c} 1 \\ 1 \\ 0 \end{array}\right)\) and
\(\vec{M}= \left(\begin{array}{c} 0 \\ 1 \\ 1 \end{array}\right)\)

we get the following:

\noindent\sphinxincludegraphics{{vis_loads}.png}


\subsection{Element Loads}
\label{\detokenize{making_a_model:element-loads}}
Element load or distributed load is understood to be a load that is distributed across a connecting element.
In Beamon such loads are constantly distributed across elements.
Element loads can be only defined using an existing element index number and the following four parameters:
* \(q_x\) force in local x\sphinxhyphen{}axis direction
* \(q_y\) force in local y\sphinxhyphen{}axis direction
* \(q_z\) force in local z\sphinxhyphen{}axis direction
* \(q_w\) momentum in local x\sphinxhyphen{}axis direction


\subsubsection{With Geometry File}
\label{\detokenize{making_a_model:id11}}
To define element loads use the keyword ‘{\color{red}\bfseries{}*}lineload’. Each entry starts with a valid node index number followed by 6
floating\sphinxhyphen{}point numbers for force and momentum vector respectively.

The following syntax diagram describes element load entries.

\begin{figure}[htbp]
\centering
\capstart

\noindent\sphinxincludegraphics{{lineloadEntry}.png}
\caption{Syntax Diagram of element loads entry in geometry files}\label{\detokenize{making_a_model:id21}}\end{figure}

\sphinxstylestrong{Example}:

\begin{sphinxVerbatim}[commandchars=\\\{\}]
\PYG{o}{*}\PYG{n}{lineload}
\PYG{l+m+mi}{2}\PYG{p}{,}\PYG{l+m+mi}{0}\PYG{p}{,}\PYG{l+m+mi}{1}\PYG{p}{,}\PYG{o}{\PYGZhy{}}\PYG{l+m+mf}{7.5e+07}\PYG{p}{,}\PYG{l+m+mi}{0}
\PYG{l+m+mi}{1}\PYG{p}{,}\PYG{l+m+mi}{1}\PYG{p}{,}\PYG{l+m+mi}{0}\PYG{p}{,}\PYG{l+m+mi}{0}\PYG{p}{,}\PYG{l+m+mi}{0}
\PYG{l+m+mi}{3}\PYG{p}{,}\PYG{l+m+mi}{0}\PYG{p}{,}\PYG{l+m+mi}{0}\PYG{p}{,}\PYG{l+m+mi}{0}\PYG{p}{,}\PYG{l+m+mi}{1}
\end{sphinxVerbatim}

first element load entry has constant \(q_z=-7.5e+06\) value on element 2.


\subsubsection{With User Interface}
\label{\detokenize{making_a_model:id14}}
Analogous to {\hyperref[\detokenize{making_a_model:nodes-input}]{\sphinxcrossref{\DUrole{std,std-ref}{Nodes And Boundry Conditions}}}}


\subsubsection{Element Loads Visualization}
\label{\detokenize{making_a_model:element-loads-visualization}}
From example above we get following visualization

\noindent\sphinxincludegraphics{{vis_lineloads}.png}


\subsection{Profiles and BeamSize}
\label{\detokenize{making_a_model:profiles-and-beamsize}}
Simulating 3D Beams requires specific element properties (profiles). Each profile must contain following parameters:
\begin{itemize}
\item {} 
\(E\): modulus of elasticity E

\item {} 
\(G\): shear modulus

\item {} 
\(A\): cross section area

\item {} 
\(Iy\): moment of inertia with respect to the local y\sphinxhyphen{}axis

\item {} 
\(Iz\): moment of inertia with respect to the local z\sphinxhyphen{}axis

\item {} 
\(Kv\): St Venant torsional stiffness

\item {} 
\(K\): optional spring stiffness (excluded from simulation momentarily)

\end{itemize}

\sphinxstylestrong{Note}: It is assumed that \(I_{yz}\) is zero.


\subsubsection{With Geometry File}
\label{\detokenize{making_a_model:id15}}
To define profiles use keyword ‘*profile’.
The following syntax diagram shows how each profile entry should be.

\begin{figure}[htbp]
\centering
\capstart

\noindent\sphinxincludegraphics{{profileEntry}.png}
\caption{Syntax diagram for profile entries in geometry files}\label{\detokenize{making_a_model:id22}}\end{figure}

\sphinxstylestrong{Example}:

\begin{sphinxVerbatim}[commandchars=\\\{\}]
\PYG{o}{*}\PYG{n}{profile}
\PYG{l+m+mf}{7e+10}\PYG{p}{,}\PYG{l+m+mf}{3e+10}\PYG{p}{,}\PYG{l+m+mf}{0.25}\PYG{p}{,}\PYG{l+m+mf}{0.00520833}\PYG{p}{,}\PYG{l+m+mf}{0.00520833}\PYG{p}{,}\PYG{l+m+mf}{0.00880208}\PYG{p}{,}\PYG{l+m+mi}{0}
\end{sphinxVerbatim}


\subsubsection{With User Interface}
\label{\detokenize{making_a_model:id16}}
Analogous to {\hyperref[\detokenize{making_a_model:nodes-input}]{\sphinxcrossref{\DUrole{std,std-ref}{Nodes And Boundry Conditions}}}}


\subsubsection{With BeamSize}
\label{\detokenize{making_a_model:with-beamsize}}
Calculating profile values could be tricky, especially if you are dealing with complex profiles.
\sphinxstylestrong{BeamSize} should simplify calculating profile values depending on which geometry you use.

It is worth noting that in the current development only profiles with \(I_{yz}=0\) yield correct simulation results.

Steps to calculate profiles:
\begin{enumerate}
\sphinxsetlistlabels{\arabic}{enumi}{enumii}{}{.}%
\item {} 
Choose geometry type (one of the tabs)

\item {} 
Enter profile dimensions in “Dimensions” section or just click and drag one of the handles in the drawing on the right side

\item {} 
Results in “Output” section can be saved by clicking “save” button

\item {} 
Save results in your project by entering modulus of elasticity and shear modulus.

\item {} 
Calculated and entered values can be seen in profiles table in the visualizer.

\end{enumerate}

\begin{figure}[htbp]
\centering
\capstart

\noindent\sphinxincludegraphics{{beamsize}.png}
\caption{BeamSize GUI depicting interaction with drawing.}\label{\detokenize{making_a_model:id23}}\end{figure}


\section{Theory}
\label{\detokenize{theory:theory}}\label{\detokenize{theory::doc}}
This project is base on bachelor thesis “Entwicklung eines Softwaresystems zur Visualisierung und Simulation von
Tragwerksstrukturen” in english “Development of a software system for the visualization and simulation of truss
structures” at institute of structural mechanics and lightweight design at RWTH\sphinxhyphen{}Aachen University.


\subsection{Element Orientation}
\label{\detokenize{theory:element-orientation}}
Supposedly we have an element defined between points A and B.
We define elements local z\sphinxhyphen{}axis by an auxiliary vector \(\vec{v}\). As shown in Figure below, \(\vec{v}\) has one
orthogonal projection on the normal plane E. The normal vector of E points in the direction
of local x\sphinxhyphen{}axis. This projection of \(\vec{v}\) onto plane E points in the direction of the local z\sphinxhyphen{}axis.

\begin{figure}[htbp]
\centering
\capstart

\noindent\sphinxincludegraphics{{DefiningBeamOrientation}.png}
\caption{The rotation of an element is based only on the local z\sphinxhyphen{}axis, which is defined by the vector \(\vec{v}\)}\label{\detokenize{theory:id1}}\end{figure}


\subsection{CALFEM}
\label{\detokenize{theory:calfem}}
Beamon’s simulation is based on open source library CALFEM. For further information read
\sphinxcode{\sphinxupquote{CALFEM for Matlab}} and
\sphinxcode{\sphinxupquote{CALFEM for Python}}


\section{API}
\label{\detokenize{api:api}}\label{\detokenize{api::doc}}

\subsection{Core}
\label{\detokenize{api:module-beamon.core}}\label{\detokenize{api:core}}\index{module@\spxentry{module}!beamon.core@\spxentry{beamon.core}}\index{beamon.core@\spxentry{beamon.core}!module@\spxentry{module}}
Core functions for beamon
\index{assemble\_bc() (in module beamon.core)@\spxentry{assemble\_bc()}\spxextra{in module beamon.core}}

\begin{fulllineitems}
\phantomsection\label{\detokenize{api:beamon.core.assemble_bc}}\pysiglinewithargsret{\sphinxcode{\sphinxupquote{beamon.core.}}\sphinxbfcode{\sphinxupquote{assemble\_bc}}}{\emph{\DUrole{n}{nodes\_bc}}, \emph{\DUrole{n}{Dof}}}{}
Assemble boundary condition information in a vector.
nodes\_bc should contain {[}u\_x,u\_y,u\_z,phi\_x,phi\_y,phi\_z{]} for each node, which can only contain
0 (free) or 1 (locked).
The assembled bc vector could be ex. {[}1,2,5,6{]} that locks movement in those numbered directions.
\begin{quote}\begin{description}
\item[{Parameters}] \leavevmode\begin{itemize}
\item {} 
\sphinxstyleliteralstrong{\sphinxupquote{nodes\_bc}} (\sphinxstyleliteralemphasis{\sphinxupquote{integer matrix}}) \textendash{} boundary condition information from database

\item {} 
\sphinxstyleliteralstrong{\sphinxupquote{Dof}} (\sphinxstyleliteralemphasis{\sphinxupquote{integer matrix}}) \textendash{} nodes degrees of freedom

\end{itemize}

\item[{Returns}] \leavevmode
bc vector

\item[{Return type}] \leavevmode
integer 1 x n vector

\end{description}\end{quote}

\end{fulllineitems}

\index{assemble\_element\_loads() (in module beamon.core)@\spxentry{assemble\_element\_loads()}\spxextra{in module beamon.core}}

\begin{fulllineitems}
\phantomsection\label{\detokenize{api:beamon.core.assemble_element_loads}}\pysiglinewithargsret{\sphinxcode{\sphinxupquote{beamon.core.}}\sphinxbfcode{\sphinxupquote{assemble\_element\_loads}}}{\emph{\DUrole{n}{loads}}, \emph{\DUrole{n}{Edof}}}{}
Assemble element local loads vector from element loads.
Loads should contain {[}link\_id, qx, qy, qz, qw{]}
\begin{quote}\begin{description}
\item[{Parameters}] \leavevmode\begin{itemize}
\item {} 
\sphinxstyleliteralstrong{\sphinxupquote{loads}} (\sphinxstyleliteralemphasis{\sphinxupquote{float matrix}}) \textendash{} element loads

\item {} 
\sphinxstyleliteralstrong{\sphinxupquote{Edof}} (\sphinxstyleliteralemphasis{\sphinxupquote{integer matrix}}) \textendash{} Element degree of freedom (Topology matrix)

\end{itemize}

\item[{Returns}] \leavevmode
local forces

\item[{Return type}] \leavevmode
float matrix

\end{description}\end{quote}

\end{fulllineitems}

\index{assemble\_global\_f() (in module beamon.core)@\spxentry{assemble\_global\_f()}\spxextra{in module beamon.core}}

\begin{fulllineitems}
\phantomsection\label{\detokenize{api:beamon.core.assemble_global_f}}\pysiglinewithargsret{\sphinxcode{\sphinxupquote{beamon.core.}}\sphinxbfcode{\sphinxupquote{assemble\_global\_f}}}{\emph{\DUrole{n}{loads}}, \emph{\DUrole{n}{Dof}}}{}
Assemble global loads vector from nodes loads.
Loads should contain nodes coordinates and the forces applied to each node {[}x,y,z, u,v,w, m\_x,m\_y,m\_z{]}
\begin{quote}\begin{description}
\item[{Parameters}] \leavevmode\begin{itemize}
\item {} 
\sphinxstyleliteralstrong{\sphinxupquote{loads}} (\sphinxstyleliteralemphasis{\sphinxupquote{float matrix}}) \textendash{} Nodes loads array

\item {} 
\sphinxstyleliteralstrong{\sphinxupquote{Dof}} (\sphinxstyleliteralemphasis{\sphinxupquote{integer matrix}}) \textendash{} nodes degrees of freedom

\end{itemize}

\item[{Returns}] \leavevmode
global force vector (n x 1 vector)

\item[{Return type}] \leavevmode
float vector

\end{description}\end{quote}

\end{fulllineitems}

\index{assemble\_nodes\_results() (in module beamon.core)@\spxentry{assemble\_nodes\_results()}\spxextra{in module beamon.core}}

\begin{fulllineitems}
\phantomsection\label{\detokenize{api:beamon.core.assemble_nodes_results}}\pysiglinewithargsret{\sphinxcode{\sphinxupquote{beamon.core.}}\sphinxbfcode{\sphinxupquote{assemble\_nodes\_results}}}{\emph{\DUrole{n}{o\_edof}}, \emph{\DUrole{n}{nodes}}, \emph{\DUrole{n}{a}}, \emph{\DUrole{n}{r}}, \emph{\DUrole{n}{map}}}{}
Assemble nodes displacements and support forces according to joints mapping into a pandas dataframe.
\begin{quote}\begin{description}
\item[{Parameters}] \leavevmode\begin{itemize}
\item {} 
\sphinxstyleliteralstrong{\sphinxupquote{o\_edof}} (\sphinxstyleliteralemphasis{\sphinxupquote{integer matrix}}) \textendash{} origin edof

\item {} 
\sphinxstyleliteralstrong{\sphinxupquote{nodes}} (\sphinxstyleliteralemphasis{\sphinxupquote{float matrix}}) \textendash{} nodes coordinates

\item {} 
\sphinxstyleliteralstrong{\sphinxupquote{a}} (\sphinxstyleliteralemphasis{\sphinxupquote{float vector}}) \textendash{} nodes displacements

\item {} 
\sphinxstyleliteralstrong{\sphinxupquote{r}} (\sphinxstyleliteralemphasis{\sphinxupquote{float vector}}) \textendash{} nodes support forces

\item {} 
\sphinxstyleliteralstrong{\sphinxupquote{map}} (\sphinxhref{https://docs.python.org/3/library/stdtypes.html\#dict}{\sphinxstyleliteralemphasis{\sphinxupquote{dict}}}) \textendash{} origin nodes to cloned joint nodes map

\end{itemize}

\end{description}\end{quote}

\end{fulllineitems}

\index{get\_local\_orientation() (in module beamon.core)@\spxentry{get\_local\_orientation()}\spxextra{in module beamon.core}}

\begin{fulllineitems}
\phantomsection\label{\detokenize{api:beamon.core.get_local_orientation}}\pysiglinewithargsret{\sphinxcode{\sphinxupquote{beamon.core.}}\sphinxbfcode{\sphinxupquote{get\_local\_orientation}}}{\emph{\DUrole{n}{orientation}}}{}~\begin{description}
\item[{Calculates all local axes orientation vectors according to v vector and}] \leavevmode
normal vectors n = (node2\sphinxhyphen{}node1)/abs(node2\sphinxhyphen{}node1)

\end{description}

Z axes is the projection of v on the plane with orthogonal vector X axes
\begin{quote}\begin{description}
\item[{Returns}] \leavevmode
x,y,z directions in each row

\end{description}\end{quote}

\end{fulllineitems}

\index{get\_rmat() (in module beamon.core)@\spxentry{get\_rmat()}\spxextra{in module beamon.core}}

\begin{fulllineitems}
\phantomsection\label{\detokenize{api:beamon.core.get_rmat}}\pysiglinewithargsret{\sphinxcode{\sphinxupquote{beamon.core.}}\sphinxbfcode{\sphinxupquote{get\_rmat}}}{\emph{\DUrole{n}{M}}, \emph{\DUrole{n}{N}}}{}
Gets the 3D Rotation matrix of a plane defined with vector normal M.
N will be the vector normal to the plane you want to rotate into.
\begin{quote}\begin{description}
\item[{Parameters}] \leavevmode\begin{itemize}
\item {} 
\sphinxstyleliteralstrong{\sphinxupquote{M}} (\sphinxstyleliteralemphasis{\sphinxupquote{double}}) \textendash{} 3 component list

\item {} 
\sphinxstyleliteralstrong{\sphinxupquote{N}} (\sphinxstyleliteralemphasis{\sphinxupquote{double}}) \textendash{} 3 component list

\end{itemize}

\item[{Returns}] \leavevmode
3x3 matrix

\item[{Return type}] \leavevmode
double

\end{description}\end{quote}

\end{fulllineitems}

\index{make\_joints() (in module beamon.core)@\spxentry{make\_joints()}\spxextra{in module beamon.core}}

\begin{fulllineitems}
\phantomsection\label{\detokenize{api:beamon.core.make_joints}}\pysiglinewithargsret{\sphinxcode{\sphinxupquote{beamon.core.}}\sphinxbfcode{\sphinxupquote{make\_joints}}}{\emph{\DUrole{n}{edof}}, \emph{\DUrole{n}{nodes}}, \emph{\DUrole{n}{dof}}, \emph{\DUrole{n}{ndofs}}, \emph{\DUrole{n}{bc}}}{}
Extends edof, nodes, dof and bc matrices with new cloned nodes that has new dof.
This will adjust translation conditions between elements.
New nodes and dofs will be appended at the ende of each input ‘nodes’ and ‘dof’.
edof matrix contain element dof numbers for each nodal point end. Free dof number is predefined as zero.
nodes matrix contain {[}x,y,z{]} nodes coordinates in each row.
dof matrix contain nodes degrees of freedom for each node in a row.
bc array contain nodes degrees of freedom that are locked.
map contains mapping information from cloned nodes to origin nodes
\begin{quote}\begin{description}
\item[{Parameters}] \leavevmode\begin{itemize}
\item {} 
\sphinxstyleliteralstrong{\sphinxupquote{edof}} (\sphinxstyleliteralemphasis{\sphinxupquote{integer matrix}}) \textendash{} topology matrix

\item {} 
\sphinxstyleliteralstrong{\sphinxupquote{nodes}} (\sphinxstyleliteralemphasis{\sphinxupquote{float matrix}}) \textendash{} nodes coordinates matrix

\item {} 
\sphinxstyleliteralstrong{\sphinxupquote{dof}} (\sphinxstyleliteralemphasis{\sphinxupquote{integer matrix}}) \textendash{} nodes dof matrix

\item {} 
\sphinxstyleliteralstrong{\sphinxupquote{ndofs}} (\sphinxstyleliteralemphasis{\sphinxupquote{integer}}\sphinxstyleliteralemphasis{\sphinxupquote{ (}}\sphinxstyleliteralemphasis{\sphinxupquote{3}}\sphinxstyleliteralemphasis{\sphinxupquote{ or }}\sphinxstyleliteralemphasis{\sphinxupquote{6}}\sphinxstyleliteralemphasis{\sphinxupquote{)}}) \textendash{} number of dof in each node

\item {} 
\sphinxstyleliteralstrong{\sphinxupquote{bc}} (\sphinxstyleliteralemphasis{\sphinxupquote{integer array}}) \textendash{} boundary conditions array

\item {} 
\sphinxstyleliteralstrong{\sphinxupquote{map}} (\sphinxhref{https://docs.python.org/3/library/stdtypes.html\#dict}{\sphinxstyleliteralemphasis{\sphinxupquote{dict}}}) \textendash{} origin and cloned nodes mapping

\end{itemize}

\item[{Returns}] \leavevmode
edof, nodes, dof, map

\end{description}\end{quote}

\end{fulllineitems}

\index{transform() (in module beamon.core)@\spxentry{transform()}\spxextra{in module beamon.core}}

\begin{fulllineitems}
\phantomsection\label{\detokenize{api:beamon.core.transform}}\pysiglinewithargsret{\sphinxcode{\sphinxupquote{beamon.core.}}\sphinxbfcode{\sphinxupquote{transform}}}{\emph{\DUrole{n}{T}}, \emph{\DUrole{n}{points}}}{}
Transforms each point in points matrix using the transformation matrix T.
points should have row wise point coordinates.
\begin{quote}\begin{description}
\item[{Parameters}] \leavevmode\begin{itemize}
\item {} 
\sphinxstyleliteralstrong{\sphinxupquote{T}} (\sphinxstyleliteralemphasis{\sphinxupquote{double}}) \textendash{} 3x3 Transformation Matrix

\item {} 
\sphinxstyleliteralstrong{\sphinxupquote{points}} (\sphinxstyleliteralemphasis{\sphinxupquote{double}}) \textendash{} nx3 points matrix

\end{itemize}

\item[{Returns}] \leavevmode
nx3 points matrix

\item[{Return type}] \leavevmode
double

\end{description}\end{quote}

\end{fulllineitems}



\subsection{Simulation}
\label{\detokenize{api:module-beamon.simulation}}\label{\detokenize{api:simulation}}\index{module@\spxentry{module}!beamon.simulation@\spxentry{beamon.simulation}}\index{beamon.simulation@\spxentry{beamon.simulation}!module@\spxentry{module}}\index{Simulation (class in beamon.simulation)@\spxentry{Simulation}\spxextra{class in beamon.simulation}}

\begin{fulllineitems}
\phantomsection\label{\detokenize{api:beamon.simulation.Simulation}}\pysiglinewithargsret{\sphinxbfcode{\sphinxupquote{class }}\sphinxcode{\sphinxupquote{beamon.simulation.}}\sphinxbfcode{\sphinxupquote{Simulation}}}{\emph{\DUrole{n}{database}}}{}
Main simulation module. Uses core and database to do calculations.
\index{element\_stats() (beamon.simulation.Simulation method)@\spxentry{element\_stats()}\spxextra{beamon.simulation.Simulation method}}

\begin{fulllineitems}
\phantomsection\label{\detokenize{api:beamon.simulation.Simulation.element_stats}}\pysiglinewithargsret{\sphinxbfcode{\sphinxupquote{element\_stats}}}{}{}
Statistics about Simulation results for elements as pandas Dataframe with following columns:
{[}‘Element Nr.’, ‘Min N’, ‘Max N’, ‘Min Vy’, ‘Max Vy’, ‘Min Vz’, ‘Max Vz’, ‘Min T’, ‘Max T’, ‘Min My’, ‘Max My’,
‘Min Mz’, ‘Max Mz’{]}
\begin{quote}\begin{description}
\item[{Returns}] \leavevmode
Simulation results statistics matrix

\item[{Return type}] \leavevmode
pandas Dataframe

\end{description}\end{quote}

\end{fulllineitems}

\index{global\_element\_displaced\_points() (beamon.simulation.Simulation method)@\spxentry{global\_element\_displaced\_points()}\spxextra{beamon.simulation.Simulation method}}

\begin{fulllineitems}
\phantomsection\label{\detokenize{api:beamon.simulation.Simulation.global_element_displaced_points}}\pysiglinewithargsret{\sphinxbfcode{\sphinxupquote{global\_element\_displaced\_points}}}{}{}
Returns displaced points coordinates in global coordinate system for each evaluation point on elements
as pandas Dataframe with following columns:
{[}‘Element Nr.’, ‘x’, ‘y’, ‘z’{]}
\begin{quote}\begin{description}
\item[{Returns}] \leavevmode
global element evaluation points matrix

\item[{Return type}] \leavevmode
pandas Dataframe

\end{description}\end{quote}

\end{fulllineitems}

\index{global\_element\_evaluation\_points() (beamon.simulation.Simulation method)@\spxentry{global\_element\_evaluation\_points()}\spxextra{beamon.simulation.Simulation method}}

\begin{fulllineitems}
\phantomsection\label{\detokenize{api:beamon.simulation.Simulation.global_element_evaluation_points}}\pysiglinewithargsret{\sphinxbfcode{\sphinxupquote{global\_element\_evaluation\_points}}}{}{}
Returns simulation evaluation point coordinates in global coordinate system
as pandas Dataframe with following columns:
{[}‘Element Nr.’, ‘x’, ‘y’, ‘z’{]}
\begin{quote}\begin{description}
\item[{Returns}] \leavevmode
global element evaluation points matrix

\item[{Return type}] \leavevmode
pandas Dataframe

\end{description}\end{quote}

\end{fulllineitems}

\index{local\_element\_displacements() (beamon.simulation.Simulation method)@\spxentry{local\_element\_displacements()}\spxextra{beamon.simulation.Simulation method}}

\begin{fulllineitems}
\phantomsection\label{\detokenize{api:beamon.simulation.Simulation.local_element_displacements}}\pysiglinewithargsret{\sphinxbfcode{\sphinxupquote{local\_element\_displacements}}}{}{}
Simulation results for local element displacements as pandas Dataframe with following columns:
{[}‘Element Nr.’, ‘xi’, ‘u’, ‘v’, ‘w’, ‘phi’{]}
\begin{quote}\begin{description}
\item[{Returns}] \leavevmode
local element displacements matrix

\item[{Return type}] \leavevmode
pandas Dataframe

\end{description}\end{quote}

\end{fulllineitems}

\index{local\_element\_section\_forces() (beamon.simulation.Simulation method)@\spxentry{local\_element\_section\_forces()}\spxextra{beamon.simulation.Simulation method}}

\begin{fulllineitems}
\phantomsection\label{\detokenize{api:beamon.simulation.Simulation.local_element_section_forces}}\pysiglinewithargsret{\sphinxbfcode{\sphinxupquote{local\_element\_section\_forces}}}{}{}
Simulation results for local element section forces as pandas Dataframe with following columns:
{[}‘Element Nr.’, ‘xi’, ‘N’, ‘Vy’, ‘Vz’, ‘T’, ‘My’,’Mz’{]}
\begin{quote}\begin{description}
\item[{Returns}] \leavevmode
local element section forces matrix

\item[{Return type}] \leavevmode
pandas Dataframe

\end{description}\end{quote}

\end{fulllineitems}

\index{nodes\_results() (beamon.simulation.Simulation method)@\spxentry{nodes\_results()}\spxextra{beamon.simulation.Simulation method}}

\begin{fulllineitems}
\phantomsection\label{\detokenize{api:beamon.simulation.Simulation.nodes_results}}\pysiglinewithargsret{\sphinxbfcode{\sphinxupquote{nodes\_results}}}{}{}
Simulation results for global nodes displacements (ax,ay,az,aphi\_x,aphi\_y,a\_phiz) with nodes coordinates
(x,y,z) and normal forces (Qx,Qy,Qz,Qphi\_x,Qphi\_y,Qphi\_z) as pandas Dataframe with following columns:
{[}‘Node Nr.’, ‘ax’, ‘ay’, ‘az’, ‘aphi\_x’, ‘aphi\_y’, ‘aphi\_z’, ‘x’, ‘y’, ‘z’, ‘Qx’, ‘Qy’, ‘Qz’, ‘Qphi\_x’,
‘Qphi\_y’, ‘Qphi\_z’{]}
\begin{quote}\begin{description}
\item[{Returns}] \leavevmode
local element displacements matrix

\item[{Return type}] \leavevmode
pandas Dataframe

\end{description}\end{quote}

\end{fulllineitems}

\index{simulate() (beamon.simulation.Simulation method)@\spxentry{simulate()}\spxextra{beamon.simulation.Simulation method}}

\begin{fulllineitems}
\phantomsection\label{\detokenize{api:beamon.simulation.Simulation.simulate}}\pysiglinewithargsret{\sphinxbfcode{\sphinxupquote{simulate}}}{\emph{\DUrole{n}{n}\DUrole{o}{=}\DUrole{default_value}{20}}, \emph{\DUrole{n}{scale}\DUrole{o}{=}\DUrole{default_value}{2}}}{}
Run the simulation according to data in the database.
\sphinxstylestrong{Warning}: Elements with no defined profiles will cause an error
\begin{quote}\begin{description}
\item[{Parameters}] \leavevmode
\sphinxstyleliteralstrong{\sphinxupquote{n}} (\sphinxstyleliteralemphasis{\sphinxupquote{integer}}) \textendash{} number of evaluation points

\end{description}\end{quote}

\end{fulllineitems}


\end{fulllineitems}



\subsection{Database}
\label{\detokenize{api:module-beamon.database}}\label{\detokenize{api:database}}\index{module@\spxentry{module}!beamon.database@\spxentry{beamon.database}}\index{beamon.database@\spxentry{beamon.database}!module@\spxentry{module}}\index{Database (class in beamon.database)@\spxentry{Database}\spxextra{class in beamon.database}}

\begin{fulllineitems}
\phantomsection\label{\detokenize{api:beamon.database.Database}}\pysiglinewithargsret{\sphinxbfcode{\sphinxupquote{class }}\sphinxcode{\sphinxupquote{beamon.database.}}\sphinxbfcode{\sphinxupquote{Database}}}{\emph{\DUrole{n}{path\_to\_database}\DUrole{o}{=}\DUrole{default_value}{None}}, \emph{\DUrole{n}{check}\DUrole{o}{=}\DUrole{default_value}{True}}, \emph{\DUrole{n}{ram}\DUrole{o}{=}\DUrole{default_value}{False}}, \emph{\DUrole{n}{path\_to\_save\_database}\DUrole{o}{=}\DUrole{default_value}{\textquotesingle{}\textquotesingle{}}}}{}
A data set to save all the data for constructing Elements in Visualization and for calculations in Beamon
\sphinxstylestrong{Warning}: This class should follow singleton pattern.

conn: SQLite3 connection object

cur: cursor for SQLite queries
\index{add\_lineload() (beamon.database.Database method)@\spxentry{add\_lineload()}\spxextra{beamon.database.Database method}}

\begin{fulllineitems}
\phantomsection\label{\detokenize{api:beamon.database.Database.add_lineload}}\pysiglinewithargsret{\sphinxbfcode{\sphinxupquote{add\_lineload}}}{\emph{\DUrole{n}{l\_index}}, \emph{\DUrole{n}{qx}}, \emph{\DUrole{n}{qy}}, \emph{\DUrole{n}{qz}}, \emph{\DUrole{n}{qw}}}{}
Add a load to lineload table.
\begin{quote}\begin{description}
\item[{Parameters}] \leavevmode\begin{itemize}
\item {} 
\sphinxstyleliteralstrong{\sphinxupquote{l\_index}} (\sphinxstyleliteralemphasis{\sphinxupquote{integer}}) \textendash{} link index number

\item {} 
\sphinxstyleliteralstrong{\sphinxupquote{qx}} (\sphinxhref{https://docs.python.org/3/library/functions.html\#float}{\sphinxstyleliteralemphasis{\sphinxupquote{float}}}) \textendash{} constant force in x direction

\item {} 
\sphinxstyleliteralstrong{\sphinxupquote{qy}} (\sphinxhref{https://docs.python.org/3/library/functions.html\#float}{\sphinxstyleliteralemphasis{\sphinxupquote{float}}}) \textendash{} constant force in y direction

\item {} 
\sphinxstyleliteralstrong{\sphinxupquote{qz}} (\sphinxhref{https://docs.python.org/3/library/functions.html\#float}{\sphinxstyleliteralemphasis{\sphinxupquote{float}}}) \textendash{} constant force in z direction

\item {} 
\sphinxstyleliteralstrong{\sphinxupquote{qw}} (\sphinxhref{https://docs.python.org/3/library/functions.html\#float}{\sphinxstyleliteralemphasis{\sphinxupquote{float}}}) \textendash{} constant momentum in x direction

\end{itemize}

\item[{Returns}] \leavevmode
True

\end{description}\end{quote}

\end{fulllineitems}

\index{add\_link() (beamon.database.Database method)@\spxentry{add\_link()}\spxextra{beamon.database.Database method}}

\begin{fulllineitems}
\phantomsection\label{\detokenize{api:beamon.database.Database.add_link}}\pysiglinewithargsret{\sphinxbfcode{\sphinxupquote{add\_link}}}{\emph{\DUrole{n}{index1}}, \emph{\DUrole{n}{index2}}, \emph{\DUrole{n}{dof}\DUrole{o}{=}\DUrole{default_value}{None}}, \emph{\DUrole{n}{profile\_id}\DUrole{o}{=}\DUrole{default_value}{None}}, \emph{\DUrole{n}{v\_x}\DUrole{o}{=}\DUrole{default_value}{None}}, \emph{\DUrole{n}{v\_y}\DUrole{o}{=}\DUrole{default_value}{None}}, \emph{\DUrole{n}{v\_z}\DUrole{o}{=}\DUrole{default_value}{None}}}{}
Addes a link from node with index 1 to node with index2 and profile with profile\_id
if append is False a Link with specified values will be added.
\begin{quote}\begin{description}
\item[{Parameters}] \leavevmode\begin{itemize}
\item {} 
\sphinxstyleliteralstrong{\sphinxupquote{index1}} \textendash{} starting node index

\item {} 
\sphinxstyleliteralstrong{\sphinxupquote{index2}} \textendash{} ending node index

\item {} 
\sphinxstyleliteralstrong{\sphinxupquote{dof}} \textendash{} degrees of freedom {[}ux1, uy1, uz1, phix1, … , uz2, phix2, phiy2, phiz2{]}

\item {} 
\sphinxstyleliteralstrong{\sphinxupquote{profile\_id}} \textendash{} optional index of the profile

\item {} 
\sphinxstyleliteralstrong{\sphinxupquote{append}} \textendash{} default True. Decide if the given Values should be appended to the database or not

\item {} 
\sphinxstyleliteralstrong{\sphinxupquote{v\_x}} \textendash{} x component of the direction determiner of the z local axes

\item {} 
\sphinxstyleliteralstrong{\sphinxupquote{v\_y}} \textendash{} y component of the direction determiner of the z local axes

\item {} 
\sphinxstyleliteralstrong{\sphinxupquote{v\_z}} \textendash{} z component of the direction determiner of the z local axes

\end{itemize}

\item[{Returns}] \leavevmode
True: if successfully added a link, False: if one of the indexes don’t exist

\end{description}\end{quote}

or the profile index don’t exist

\end{fulllineitems}

\index{add\_load() (beamon.database.Database method)@\spxentry{add\_load()}\spxextra{beamon.database.Database method}}

\begin{fulllineitems}
\phantomsection\label{\detokenize{api:beamon.database.Database.add_load}}\pysiglinewithargsret{\sphinxbfcode{\sphinxupquote{add\_load}}}{\emph{\DUrole{n}{index}}, \emph{\DUrole{n}{x}}, \emph{\DUrole{n}{y}}, \emph{\DUrole{n}{z}}, \emph{\DUrole{n}{m\_x}}, \emph{\DUrole{n}{m\_y}}, \emph{\DUrole{n}{m\_z}}, \emph{\DUrole{n}{lineload\_id}\DUrole{o}{=}\DUrole{default_value}{None}}}{}
adds a load on the node with index
\begin{quote}\begin{description}
\item[{Parameters}] \leavevmode\begin{itemize}
\item {} 
\sphinxstyleliteralstrong{\sphinxupquote{index}} (\sphinxstyleliteralemphasis{\sphinxupquote{integer}}) \textendash{} node index number

\item {} 
\sphinxstyleliteralstrong{\sphinxupquote{x}} (\sphinxhref{https://docs.python.org/3/library/functions.html\#float}{\sphinxstyleliteralemphasis{\sphinxupquote{float}}}) \textendash{} first component of the force vector

\item {} 
\sphinxstyleliteralstrong{\sphinxupquote{y}} (\sphinxhref{https://docs.python.org/3/library/functions.html\#float}{\sphinxstyleliteralemphasis{\sphinxupquote{float}}}) \textendash{} second component of the force vector

\item {} 
\sphinxstyleliteralstrong{\sphinxupquote{z}} (\sphinxhref{https://docs.python.org/3/library/functions.html\#float}{\sphinxstyleliteralemphasis{\sphinxupquote{float}}}) \textendash{} third component of the force vector

\item {} 
\sphinxstyleliteralstrong{\sphinxupquote{m\_x}} (\sphinxhref{https://docs.python.org/3/library/functions.html\#float}{\sphinxstyleliteralemphasis{\sphinxupquote{float}}}) \textendash{} first component of the momentum vector

\item {} 
\sphinxstyleliteralstrong{\sphinxupquote{m\_y}} (\sphinxhref{https://docs.python.org/3/library/functions.html\#float}{\sphinxstyleliteralemphasis{\sphinxupquote{float}}}) \textendash{} second component of the momentum vector

\item {} 
\sphinxstyleliteralstrong{\sphinxupquote{m\_z}} (\sphinxhref{https://docs.python.org/3/library/functions.html\#float}{\sphinxstyleliteralemphasis{\sphinxupquote{float}}}) \textendash{} third component of the momentum vector

\item {} 
\sphinxstyleliteralstrong{\sphinxupquote{lineload\_id}} (\sphinxstyleliteralemphasis{\sphinxupquote{integer}}) \textendash{} optional number of reference lineload

\end{itemize}

\item[{Returns}] \leavevmode
True: if successfully added or updated / False: if node doesnt exist

\item[{Return type}] \leavevmode
boolean

\end{description}\end{quote}

\end{fulllineitems}

\index{add\_node() (beamon.database.Database method)@\spxentry{add\_node()}\spxextra{beamon.database.Database method}}

\begin{fulllineitems}
\phantomsection\label{\detokenize{api:beamon.database.Database.add_node}}\pysiglinewithargsret{\sphinxbfcode{\sphinxupquote{add\_node}}}{\emph{\DUrole{n}{x}}, \emph{\DUrole{n}{y}}, \emph{\DUrole{n}{z}}, \emph{\DUrole{n}{u\_x}\DUrole{o}{=}\DUrole{default_value}{1}}, \emph{\DUrole{n}{u\_y}\DUrole{o}{=}\DUrole{default_value}{1}}, \emph{\DUrole{n}{u\_z}\DUrole{o}{=}\DUrole{default_value}{1}}, \emph{\DUrole{n}{phi\_x}\DUrole{o}{=}\DUrole{default_value}{1}}, \emph{\DUrole{n}{phi\_y}\DUrole{o}{=}\DUrole{default_value}{1}}, \emph{\DUrole{n}{phi\_z}\DUrole{o}{=}\DUrole{default_value}{1}}, \emph{\DUrole{n}{n\_type}\DUrole{o}{=}\DUrole{default_value}{1}}}{}
adds a node with coordinates x,y,z and the freedom of movement for all 6 degrees of freedom (dof)
default values for {[}u\_x, .. ,phi\_z{]} is e vector {[}1,…,1{]}.
\begin{quote}\begin{description}
\item[{Return False}] \leavevmode
if node already exists, True: if node has been added

\end{description}\end{quote}

\end{fulllineitems}

\index{add\_profile() (beamon.database.Database method)@\spxentry{add\_profile()}\spxextra{beamon.database.Database method}}

\begin{fulllineitems}
\phantomsection\label{\detokenize{api:beamon.database.Database.add_profile}}\pysiglinewithargsret{\sphinxbfcode{\sphinxupquote{add\_profile}}}{\emph{\DUrole{n}{E}}, \emph{\DUrole{n}{G}}, \emph{\DUrole{n}{A}}, \emph{\DUrole{n}{Iy}}, \emph{\DUrole{n}{Iz}}, \emph{\DUrole{n}{kv}}, \emph{\DUrole{n}{k}}}{}
Adds a profile

\end{fulllineitems}

\index{change\_dof() (beamon.database.Database method)@\spxentry{change\_dof()}\spxextra{beamon.database.Database method}}

\begin{fulllineitems}
\phantomsection\label{\detokenize{api:beamon.database.Database.change_dof}}\pysiglinewithargsret{\sphinxbfcode{\sphinxupquote{change\_dof}}}{\emph{\DUrole{n}{index}}, \emph{\DUrole{n}{u\_x}\DUrole{o}{=}\DUrole{default_value}{None}}, \emph{\DUrole{n}{u\_y}\DUrole{o}{=}\DUrole{default_value}{None}}, \emph{\DUrole{n}{u\_z}\DUrole{o}{=}\DUrole{default_value}{None}}, \emph{\DUrole{n}{phi\_x}\DUrole{o}{=}\DUrole{default_value}{None}}, \emph{\DUrole{n}{phi\_y}\DUrole{o}{=}\DUrole{default_value}{None}}, \emph{\DUrole{n}{phi\_z}\DUrole{o}{=}\DUrole{default_value}{None}}}{}~\begin{description}
\item[{change the degree of freedom for a given node with specified node index. DOF specification:}] \leavevmode
1 is completely free, 0 is locked.

\end{description}

for example (index=300, u\_x=1, u\_y=1, u\_z=1, phi\_x=1, phi\_y=1, phi\_z=0) locks the rotation
in the z axes of node 300.
\begin{quote}\begin{description}
\item[{Parameters}] \leavevmode\begin{itemize}
\item {} 
\sphinxstyleliteralstrong{\sphinxupquote{index}} \textendash{} index of the node

\item {} 
\sphinxstyleliteralstrong{\sphinxupquote{u\_x}} \textendash{} movement in x direction

\item {} 
\sphinxstyleliteralstrong{\sphinxupquote{u\_y}} \textendash{} movement in y direction

\item {} 
\sphinxstyleliteralstrong{\sphinxupquote{u\_z}} \textendash{} movement in z direction

\item {} 
\sphinxstyleliteralstrong{\sphinxupquote{phi\_x}} \textendash{} rotation in x axes

\item {} 
\sphinxstyleliteralstrong{\sphinxupquote{phi\_y}} \textendash{} rotation in y axes

\item {} 
\sphinxstyleliteralstrong{\sphinxupquote{phi\_z}} \textendash{} rotation in z axes

\end{itemize}

\item[{Returns}] \leavevmode
True: if successfully changed/ False: if node not found

\end{description}\end{quote}

\end{fulllineitems}

\index{change\_edof() (beamon.database.Database method)@\spxentry{change\_edof()}\spxextra{beamon.database.Database method}}

\begin{fulllineitems}
\phantomsection\label{\detokenize{api:beamon.database.Database.change_edof}}\pysiglinewithargsret{\sphinxbfcode{\sphinxupquote{change\_edof}}}{\emph{\DUrole{n}{edof}}, \emph{\DUrole{n}{n\_index}\DUrole{o}{=}\DUrole{default_value}{None}}, \emph{\DUrole{n}{e\_index}\DUrole{o}{=}\DUrole{default_value}{None}}}{}
Changes given degrees of freedom for all elements connected to the node with index.
If e\_index is given only the degrees of freedom for that element will be changed.
\begin{quote}\begin{description}
\item[{Parameters}] \leavevmode\begin{itemize}
\item {} 
\sphinxstyleliteralstrong{\sphinxupquote{n\_index}} \textendash{} index of the node | integer

\item {} 
\sphinxstyleliteralstrong{\sphinxupquote{edof}} \textendash{} list of integers {[}ux1, uy1, uz1, phix1, … , uz2, phix2, phiy2, phiz2{]}

\item {} 
\sphinxstyleliteralstrong{\sphinxupquote{e\_index}} \textendash{} index of the element | integer

\end{itemize}

\end{description}\end{quote}

\end{fulllineitems}

\index{change\_lineload() (beamon.database.Database method)@\spxentry{change\_lineload()}\spxextra{beamon.database.Database method}}

\begin{fulllineitems}
\phantomsection\label{\detokenize{api:beamon.database.Database.change_lineload}}\pysiglinewithargsret{\sphinxbfcode{\sphinxupquote{change\_lineload}}}{\emph{\DUrole{n}{index}}, \emph{\DUrole{n}{l\_index}}, \emph{\DUrole{n}{qx}}, \emph{\DUrole{n}{qy}}, \emph{\DUrole{n}{qz}}, \emph{\DUrole{n}{qw}}}{}
Change lineload data with certain lineload index.
\begin{quote}\begin{description}
\item[{Parameters}] \leavevmode\begin{itemize}
\item {} 
\sphinxstyleliteralstrong{\sphinxupquote{index}} (\sphinxstyleliteralemphasis{\sphinxupquote{integer}}) \textendash{} lineload index number

\item {} 
\sphinxstyleliteralstrong{\sphinxupquote{l\_index}} (\sphinxstyleliteralemphasis{\sphinxupquote{integer}}) \textendash{} link index number

\item {} 
\sphinxstyleliteralstrong{\sphinxupquote{qx}} (\sphinxhref{https://docs.python.org/3/library/functions.html\#float}{\sphinxstyleliteralemphasis{\sphinxupquote{float}}}) \textendash{} constant force in x direction

\item {} 
\sphinxstyleliteralstrong{\sphinxupquote{qy}} (\sphinxhref{https://docs.python.org/3/library/functions.html\#float}{\sphinxstyleliteralemphasis{\sphinxupquote{float}}}) \textendash{} constant force in y direction

\item {} 
\sphinxstyleliteralstrong{\sphinxupquote{qz}} (\sphinxhref{https://docs.python.org/3/library/functions.html\#float}{\sphinxstyleliteralemphasis{\sphinxupquote{float}}}) \textendash{} constant force in z direction

\item {} 
\sphinxstyleliteralstrong{\sphinxupquote{qw}} (\sphinxhref{https://docs.python.org/3/library/functions.html\#float}{\sphinxstyleliteralemphasis{\sphinxupquote{float}}}) \textendash{} constant momentum in x direction

\end{itemize}

\item[{Returns}] \leavevmode
True

\end{description}\end{quote}

\end{fulllineitems}

\index{change\_link() (beamon.database.Database method)@\spxentry{change\_link()}\spxextra{beamon.database.Database method}}

\begin{fulllineitems}
\phantomsection\label{\detokenize{api:beamon.database.Database.change_link}}\pysiglinewithargsret{\sphinxbfcode{\sphinxupquote{change\_link}}}{\emph{\DUrole{n}{index}}, \emph{\DUrole{n}{n1\_id}}, \emph{\DUrole{n}{n2\_id}}, \emph{\DUrole{n}{v\_x}}, \emph{\DUrole{n}{v\_y}}, \emph{\DUrole{n}{v\_z}}, \emph{\DUrole{n}{profile\_id}}}{}
Changes specific link values with index number.
\begin{quote}\begin{description}
\item[{Returns}] \leavevmode
True/False

\end{description}\end{quote}

\end{fulllineitems}

\index{change\_load() (beamon.database.Database method)@\spxentry{change\_load()}\spxextra{beamon.database.Database method}}

\begin{fulllineitems}
\phantomsection\label{\detokenize{api:beamon.database.Database.change_load}}\pysiglinewithargsret{\sphinxbfcode{\sphinxupquote{change\_load}}}{\emph{\DUrole{n}{index}}, \emph{\DUrole{n}{n\_index}}, \emph{\DUrole{n}{x}}, \emph{\DUrole{n}{y}}, \emph{\DUrole{n}{z}}, \emph{\DUrole{n}{m\_x}}, \emph{\DUrole{n}{m\_y}}, \emph{\DUrole{n}{m\_z}}}{}
Change load information with specific index number.
\begin{quote}\begin{description}
\item[{Parameters}] \leavevmode\begin{itemize}
\item {} 
\sphinxstyleliteralstrong{\sphinxupquote{index}} (\sphinxstyleliteralemphasis{\sphinxupquote{integer}}) \textendash{} load index to be changed

\item {} 
\sphinxstyleliteralstrong{\sphinxupquote{n\_index}} (\sphinxstyleliteralemphasis{\sphinxupquote{integer}}) \textendash{} node index number

\item {} 
\sphinxstyleliteralstrong{\sphinxupquote{x}} (\sphinxhref{https://docs.python.org/3/library/functions.html\#float}{\sphinxstyleliteralemphasis{\sphinxupquote{float}}}) \textendash{} x\sphinxhyphen{}component

\item {} 
\sphinxstyleliteralstrong{\sphinxupquote{y}} (\sphinxhref{https://docs.python.org/3/library/functions.html\#float}{\sphinxstyleliteralemphasis{\sphinxupquote{float}}}) \textendash{} y\sphinxhyphen{}component

\item {} 
\sphinxstyleliteralstrong{\sphinxupquote{z}} (\sphinxhref{https://docs.python.org/3/library/functions.html\#float}{\sphinxstyleliteralemphasis{\sphinxupquote{float}}}) \textendash{} z\sphinxhyphen{}component

\item {} 
\sphinxstyleliteralstrong{\sphinxupquote{m\_x}} (\sphinxhref{https://docs.python.org/3/library/functions.html\#float}{\sphinxstyleliteralemphasis{\sphinxupquote{float}}}) \textendash{} momentum in x\sphinxhyphen{}direction

\item {} 
\sphinxstyleliteralstrong{\sphinxupquote{m\_y}} (\sphinxhref{https://docs.python.org/3/library/functions.html\#float}{\sphinxstyleliteralemphasis{\sphinxupquote{float}}}) \textendash{} momentum in y\sphinxhyphen{}direction

\item {} 
\sphinxstyleliteralstrong{\sphinxupquote{m\_z}} (\sphinxhref{https://docs.python.org/3/library/functions.html\#float}{\sphinxstyleliteralemphasis{\sphinxupquote{float}}}) \textendash{} momentum in z\sphinxhyphen{}direction

\end{itemize}

\item[{Returns}] \leavevmode
True

\end{description}\end{quote}

\end{fulllineitems}

\index{change\_node() (beamon.database.Database method)@\spxentry{change\_node()}\spxextra{beamon.database.Database method}}

\begin{fulllineitems}
\phantomsection\label{\detokenize{api:beamon.database.Database.change_node}}\pysiglinewithargsret{\sphinxbfcode{\sphinxupquote{change\_node}}}{\emph{\DUrole{n}{index}}, \emph{\DUrole{n}{x}}, \emph{\DUrole{n}{y}}, \emph{\DUrole{n}{z}}, \emph{\DUrole{n}{u\_x}\DUrole{o}{=}\DUrole{default_value}{1}}, \emph{\DUrole{n}{u\_y}\DUrole{o}{=}\DUrole{default_value}{1}}, \emph{\DUrole{n}{u\_z}\DUrole{o}{=}\DUrole{default_value}{1}}, \emph{\DUrole{n}{phi\_x}\DUrole{o}{=}\DUrole{default_value}{1}}, \emph{\DUrole{n}{phi\_y}\DUrole{o}{=}\DUrole{default_value}{1}}, \emph{\DUrole{n}{phi\_z}\DUrole{o}{=}\DUrole{default_value}{1}}}{}
Changes node information with specified index number.
\begin{quote}\begin{description}
\item[{Returns}] \leavevmode
True/False

\end{description}\end{quote}

\end{fulllineitems}

\index{change\_profile() (beamon.database.Database method)@\spxentry{change\_profile()}\spxextra{beamon.database.Database method}}

\begin{fulllineitems}
\phantomsection\label{\detokenize{api:beamon.database.Database.change_profile}}\pysiglinewithargsret{\sphinxbfcode{\sphinxupquote{change\_profile}}}{\emph{\DUrole{n}{index}}, \emph{\DUrole{n}{E}}, \emph{\DUrole{n}{G}}, \emph{\DUrole{n}{A}}, \emph{\DUrole{n}{Iy}}, \emph{\DUrole{n}{Iz}}, \emph{\DUrole{n}{kv}}, \emph{\DUrole{n}{k}}}{}
Changes certain profile values with index
\begin{quote}\begin{description}
\item[{Returns}] \leavevmode
True/False

\end{description}\end{quote}

\end{fulllineitems}

\index{check\_database() (beamon.database.Database method)@\spxentry{check\_database()}\spxextra{beamon.database.Database method}}

\begin{fulllineitems}
\phantomsection\label{\detokenize{api:beamon.database.Database.check_database}}\pysiglinewithargsret{\sphinxbfcode{\sphinxupquote{check\_database}}}{}{}
Checks if the database has already been created or fails integrity check. Also checks if the following tables
has dublicate entries: Nodes, Links
\begin{quote}\begin{description}
\item[{Returns}] \leavevmode
True/False

\end{description}\end{quote}

\end{fulllineitems}

\index{clear\_database() (beamon.database.Database method)@\spxentry{clear\_database()}\spxextra{beamon.database.Database method}}

\begin{fulllineitems}
\phantomsection\label{\detokenize{api:beamon.database.Database.clear_database}}\pysiglinewithargsret{\sphinxbfcode{\sphinxupquote{clear\_database}}}{}{}
Delete/erase all the data from the database tables

\end{fulllineitems}

\index{clear\_lineloads() (beamon.database.Database method)@\spxentry{clear\_lineloads()}\spxextra{beamon.database.Database method}}

\begin{fulllineitems}
\phantomsection\label{\detokenize{api:beamon.database.Database.clear_lineloads}}\pysiglinewithargsret{\sphinxbfcode{\sphinxupquote{clear\_lineloads}}}{}{}
Delete all the data from the table Lineload. ON DELTE CASCADE will deal with delete anomalies

\end{fulllineitems}

\index{clear\_links() (beamon.database.Database method)@\spxentry{clear\_links()}\spxextra{beamon.database.Database method}}

\begin{fulllineitems}
\phantomsection\label{\detokenize{api:beamon.database.Database.clear_links}}\pysiglinewithargsret{\sphinxbfcode{\sphinxupquote{clear\_links}}}{}{}
Delete all the data from the table Links

\end{fulllineitems}

\index{clear\_node\_loads() (beamon.database.Database method)@\spxentry{clear\_node\_loads()}\spxextra{beamon.database.Database method}}

\begin{fulllineitems}
\phantomsection\label{\detokenize{api:beamon.database.Database.clear_node_loads}}\pysiglinewithargsret{\sphinxbfcode{\sphinxupquote{clear\_node\_loads}}}{}{}
Delete all the data from the table Force

\end{fulllineitems}

\index{clear\_nodes\_table() (beamon.database.Database method)@\spxentry{clear\_nodes\_table()}\spxextra{beamon.database.Database method}}

\begin{fulllineitems}
\phantomsection\label{\detokenize{api:beamon.database.Database.clear_nodes_table}}\pysiglinewithargsret{\sphinxbfcode{\sphinxupquote{clear\_nodes\_table}}}{}{}
Delete all the data from the table Profile

\end{fulllineitems}

\index{clear\_profile\_table() (beamon.database.Database method)@\spxentry{clear\_profile\_table()}\spxextra{beamon.database.Database method}}

\begin{fulllineitems}
\phantomsection\label{\detokenize{api:beamon.database.Database.clear_profile_table}}\pysiglinewithargsret{\sphinxbfcode{\sphinxupquote{clear\_profile\_table}}}{}{}
Delete all the data from the table Profile

\end{fulllineitems}

\index{contains\_link() (beamon.database.Database method)@\spxentry{contains\_link()}\spxextra{beamon.database.Database method}}

\begin{fulllineitems}
\phantomsection\label{\detokenize{api:beamon.database.Database.contains_link}}\pysiglinewithargsret{\sphinxbfcode{\sphinxupquote{contains\_link}}}{\emph{\DUrole{n}{index1}}, \emph{\DUrole{n}{index2}\DUrole{o}{=}\DUrole{default_value}{None}}}{}
If only index1 is given: see if a link with index1 exists.

If index1 and index2 are given: see if a link with the nodes index1 and index2 exists.
\begin{quote}\begin{description}
\item[{Parameters}] \leavevmode\begin{itemize}
\item {} 
\sphinxstyleliteralstrong{\sphinxupquote{index1}} \textendash{} Integer

\item {} 
\sphinxstyleliteralstrong{\sphinxupquote{index2}} \textendash{} Integer

\end{itemize}

\item[{Returns}] \leavevmode
True/False

\end{description}\end{quote}

\end{fulllineitems}

\index{contains\_node() (beamon.database.Database method)@\spxentry{contains\_node()}\spxextra{beamon.database.Database method}}

\begin{fulllineitems}
\phantomsection\label{\detokenize{api:beamon.database.Database.contains_node}}\pysiglinewithargsret{\sphinxbfcode{\sphinxupquote{contains\_node}}}{\emph{\DUrole{n}{x}\DUrole{o}{=}\DUrole{default_value}{None}}, \emph{\DUrole{n}{y}\DUrole{o}{=}\DUrole{default_value}{None}}, \emph{\DUrole{n}{z}\DUrole{o}{=}\DUrole{default_value}{None}}, \emph{\DUrole{n}{index}\DUrole{o}{=}\DUrole{default_value}{None}}}{}
See if node with x,y,z coordinates exists. If index is given the index will be used to search
if nothing was given then has\_nodes() will be called.
\begin{quote}\begin{description}
\item[{Parameters}] \leavevmode\begin{itemize}
\item {} 
\sphinxstyleliteralstrong{\sphinxupquote{x}} \textendash{} coordinate

\item {} 
\sphinxstyleliteralstrong{\sphinxupquote{y}} \textendash{} coordinate

\item {} 
\sphinxstyleliteralstrong{\sphinxupquote{z}} \textendash{} coordinate

\item {} 
\sphinxstyleliteralstrong{\sphinxupquote{index}} \textendash{} of the node

\end{itemize}

\item[{Returns}] \leavevmode
True if node exist, False if nodes doesnt exist

\end{description}\end{quote}

\end{fulllineitems}

\index{contains\_profile() (beamon.database.Database method)@\spxentry{contains\_profile()}\spxextra{beamon.database.Database method}}

\begin{fulllineitems}
\phantomsection\label{\detokenize{api:beamon.database.Database.contains_profile}}\pysiglinewithargsret{\sphinxbfcode{\sphinxupquote{contains\_profile}}}{\emph{\DUrole{n}{index}}}{}
Check if profile with index exists.
\begin{quote}\begin{description}
\item[{Parameters}] \leavevmode
\sphinxstyleliteralstrong{\sphinxupquote{index}} (\sphinxstyleliteralemphasis{\sphinxupquote{integer}}) \textendash{} profile index number

\item[{Returns}] \leavevmode
True/False

\item[{Return type}] \leavevmode
boolean

\end{description}\end{quote}

\end{fulllineitems}

\index{erase\_lost\_profiles\_in\_links() (beamon.database.Database method)@\spxentry{erase\_lost\_profiles\_in\_links()}\spxextra{beamon.database.Database method}}

\begin{fulllineitems}
\phantomsection\label{\detokenize{api:beamon.database.Database.erase_lost_profiles_in_links}}\pysiglinewithargsret{\sphinxbfcode{\sphinxupquote{erase\_lost\_profiles\_in\_links}}}{}{}
Remove all profile foreign keys in table Links that has no reference.
This is mainly used to reset all lost foreign keys in table links after deleting a profile tha have been used
in links

\end{fulllineitems}

\index{export\_text() (beamon.database.Database method)@\spxentry{export\_text()}\spxextra{beamon.database.Database method}}

\begin{fulllineitems}
\phantomsection\label{\detokenize{api:beamon.database.Database.export_text}}\pysiglinewithargsret{\sphinxbfcode{\sphinxupquote{export\_text}}}{\emph{\DUrole{n}{path}}}{}
Exports geometry to text file
\begin{quote}\begin{description}
\item[{Parameters}] \leavevmode
\sphinxstyleliteralstrong{\sphinxupquote{path}} \textendash{} the path including the File name to be saved (ex. filename.csv)

\item[{Returns}] \leavevmode
True if saved, False if not saved

\end{description}\end{quote}

\end{fulllineitems}

\index{get\_all\_dof() (beamon.database.Database method)@\spxentry{get\_all\_dof()}\spxextra{beamon.database.Database method}}

\begin{fulllineitems}
\phantomsection\label{\detokenize{api:beamon.database.Database.get_all_dof}}\pysiglinewithargsret{\sphinxbfcode{\sphinxupquote{get\_all\_dof}}}{}{}
Gets all nodes information which has at least one dof number equals 1 (locked)
\begin{quote}\begin{description}
\item[{Returns}] \leavevmode
{[}x,y,z,u\_x,u\_y,u\_z,phi\_x,phi\_y,phi\_z{]}

\end{description}\end{quote}

\end{fulllineitems}

\index{get\_all\_lineloads() (beamon.database.Database method)@\spxentry{get\_all\_lineloads()}\spxextra{beamon.database.Database method}}

\begin{fulllineitems}
\phantomsection\label{\detokenize{api:beamon.database.Database.get_all_lineloads}}\pysiglinewithargsret{\sphinxbfcode{\sphinxupquote{get\_all\_lineloads}}}{}{}
Gets all lineloads from Lineload table.
\begin{quote}\begin{description}
\item[{Returns}] \leavevmode
{[}link\_id, qx, qy, qz, qw{]}

\item[{Return type}] \leavevmode
float matrix

\end{description}\end{quote}

\end{fulllineitems}

\index{get\_all\_links() (beamon.database.Database method)@\spxentry{get\_all\_links()}\spxextra{beamon.database.Database method}}

\begin{fulllineitems}
\phantomsection\label{\detokenize{api:beamon.database.Database.get_all_links}}\pysiglinewithargsret{\sphinxbfcode{\sphinxupquote{get\_all\_links}}}{}{}
Gets all links in the links table.
Delivers the table as it is.
\begin{quote}\begin{description}
\item[{Returns}] \leavevmode
matrix

\end{description}\end{quote}

\end{fulllineitems}

\index{get\_all\_links\_profile\_indexes() (beamon.database.Database method)@\spxentry{get\_all\_links\_profile\_indexes()}\spxextra{beamon.database.Database method}}

\begin{fulllineitems}
\phantomsection\label{\detokenize{api:beamon.database.Database.get_all_links_profile_indexes}}\pysiglinewithargsret{\sphinxbfcode{\sphinxupquote{get\_all\_links\_profile\_indexes}}}{}{}
Gets all the profiles indexes of the existing links.
\begin{quote}\begin{description}
\item[{Returns}] \leavevmode
None: if link index don’t exist. Otherwise index of the profile

\end{description}\end{quote}

\end{fulllineitems}

\index{get\_all\_nodes() (beamon.database.Database method)@\spxentry{get\_all\_nodes()}\spxextra{beamon.database.Database method}}

\begin{fulllineitems}
\phantomsection\label{\detokenize{api:beamon.database.Database.get_all_nodes}}\pysiglinewithargsret{\sphinxbfcode{\sphinxupquote{get\_all\_nodes}}}{}{}
Gets all nodes information in the DataSet. Used for the GUI\sphinxhyphen{}Table
:return: table {[}x, y, z, u\_x, u\_y, u\_z, phi\_x, phi\_y, phi\_z{]}

\end{fulllineitems}

\index{get\_bc() (beamon.database.Database method)@\spxentry{get\_bc()}\spxextra{beamon.database.Database method}}

\begin{fulllineitems}
\phantomsection\label{\detokenize{api:beamon.database.Database.get_bc}}\pysiglinewithargsret{\sphinxbfcode{\sphinxupquote{get\_bc}}}{}{}
Get only bc information from nodes.
\begin{quote}\begin{description}
\item[{Returns}] \leavevmode
{[}u\_x,u\_y,u\_z,phi\_x,phi\_y,phi\_z{]}

\end{description}\end{quote}

\end{fulllineitems}

\index{get\_default\_edof() (beamon.database.Database method)@\spxentry{get\_default\_edof()}\spxextra{beamon.database.Database method}}

\begin{fulllineitems}
\phantomsection\label{\detokenize{api:beamon.database.Database.get_default_edof}}\pysiglinewithargsret{\sphinxbfcode{\sphinxupquote{get\_default\_edof}}}{\emph{\DUrole{n}{index}}}{}
Gets default element degrees of freedom numbers for a specific link.
\begin{quote}\begin{description}
\item[{Parameters}] \leavevmode
\sphinxstyleliteralstrong{\sphinxupquote{index}} (\sphinxstyleliteralemphasis{\sphinxupquote{integer}}) \textendash{} link index number

\item[{Returns}] \leavevmode
1 x 12 array

\item[{Return type}] \leavevmode
integer array

\end{description}\end{quote}

\end{fulllineitems}

\index{get\_default\_link\_orientation() (beamon.database.Database method)@\spxentry{get\_default\_link\_orientation()}\spxextra{beamon.database.Database method}}

\begin{fulllineitems}
\phantomsection\label{\detokenize{api:beamon.database.Database.get_default_link_orientation}}\pysiglinewithargsret{\sphinxbfcode{\sphinxupquote{get\_default\_link\_orientation}}}{\emph{\DUrole{n}{index1}}, \emph{\DUrole{n}{index2}}}{}
Gets the default orientation vector of the element from starting node with index1 and ending node with index2
\begin{quote}\begin{description}
\item[{Parameters}] \leavevmode\begin{itemize}
\item {} 
\sphinxstyleliteralstrong{\sphinxupquote{index1}} \textendash{} index of the starting node

\item {} 
\sphinxstyleliteralstrong{\sphinxupquote{index2}} \textendash{} index of the ending node

\end{itemize}

\item[{Returns}] \leavevmode
{[}v\_x, v\_y, v\_z{]}

\end{description}\end{quote}

\end{fulllineitems}

\index{get\_default\_static\_edof() (beamon.database.Database method)@\spxentry{get\_default\_static\_edof()}\spxextra{beamon.database.Database method}}

\begin{fulllineitems}
\phantomsection\label{\detokenize{api:beamon.database.Database.get_default_static_edof}}\pysiglinewithargsret{\sphinxbfcode{\sphinxupquote{get\_default\_static\_edof}}}{\emph{\DUrole{n}{id1}}, \emph{\DUrole{n}{id2}}}{}
Gets the default element degrees of freedom. Default means that all degrees of freedom are locked
(stiff connections).
\begin{quote}\begin{description}
\item[{Parameters}] \leavevmode\begin{itemize}
\item {} 
\sphinxstyleliteralstrong{\sphinxupquote{id1}} \textendash{} index of starting node

\item {} 
\sphinxstyleliteralstrong{\sphinxupquote{id2}} \textendash{} index of ending node

\end{itemize}

\item[{Returns}] \leavevmode
integer list {[}ux1, uy1, uz1, phix1, … , uz2, phix2, phiy2, phiz2{]}

\end{description}\end{quote}

\end{fulllineitems}

\index{get\_dof() (beamon.database.Database method)@\spxentry{get\_dof()}\spxextra{beamon.database.Database method}}

\begin{fulllineitems}
\phantomsection\label{\detokenize{api:beamon.database.Database.get_dof}}\pysiglinewithargsret{\sphinxbfcode{\sphinxupquote{get\_dof}}}{\emph{\DUrole{n}{index}}}{}
Get the dof values of the node with index.
\begin{quote}\begin{description}
\item[{Parameters}] \leavevmode
\sphinxstyleliteralstrong{\sphinxupquote{index}} \textendash{} of the node

\item[{Returns}] \leavevmode
{[}u\_x,u\_y,u\_z,phi\_x,phi\_y,phi\_z{]}

\end{description}\end{quote}

\end{fulllineitems}

\index{get\_edof() (beamon.database.Database method)@\spxentry{get\_edof()}\spxextra{beamon.database.Database method}}

\begin{fulllineitems}
\phantomsection\label{\detokenize{api:beamon.database.Database.get_edof}}\pysiglinewithargsret{\sphinxbfcode{\sphinxupquote{get\_edof}}}{}{}
Gets all links indices and degrees of freedom from links table.
\begin{quote}\begin{description}
\item[{Returns}] \leavevmode
{[}id, ux1,uy1,uz1,phix1,phiy1,phiz1,ux2,uy2,uz2,phix2,phiy2,phiz2{]}

\item[{Return type}] \leavevmode
integer matrix

\end{description}\end{quote}

\end{fulllineitems}

\index{get\_edof\_for\_simulation() (beamon.database.Database method)@\spxentry{get\_edof\_for\_simulation()}\spxextra{beamon.database.Database method}}

\begin{fulllineitems}
\phantomsection\label{\detokenize{api:beamon.database.Database.get_edof_for_simulation}}\pysiglinewithargsret{\sphinxbfcode{\sphinxupquote{get\_edof\_for\_simulation}}}{}{}
Gets all links starting and ending nodes indices with degrees of freedom from links table.
\begin{quote}\begin{description}
\item[{Returns}] \leavevmode
{[}n1\_id, n2\_id, ux1,uy1,uz1,phix1,phiy1,phiz1,ux2,uy2,uz2,phix2,phiy2,phiz2{]}

\item[{Return type}] \leavevmode
integer matrix

\end{description}\end{quote}

\end{fulllineitems}

\index{get\_grid\_settings() (beamon.database.Database method)@\spxentry{get\_grid\_settings()}\spxextra{beamon.database.Database method}}

\begin{fulllineitems}
\phantomsection\label{\detokenize{api:beamon.database.Database.get_grid_settings}}\pysiglinewithargsret{\sphinxbfcode{\sphinxupquote{get\_grid\_settings}}}{}{}
Gets latest grid settings from table VisualizerSettings.
\begin{quote}\begin{description}
\item[{Returns}] \leavevmode
list of params / None if table is empty

\end{description}\end{quote}

\end{fulllineitems}

\index{get\_joints() (beamon.database.Database method)@\spxentry{get\_joints()}\spxextra{beamon.database.Database method}}

\begin{fulllineitems}
\phantomsection\label{\detokenize{api:beamon.database.Database.get_joints}}\pysiglinewithargsret{\sphinxbfcode{\sphinxupquote{get\_joints}}}{}{}
Get joints locations. Those are nodes coordinates + v. v is the direction of links from joints.
\begin{quote}\begin{description}
\item[{Returns}] \leavevmode
{[}x,y,z{]}

\item[{Return type}] \leavevmode
float array

\end{description}\end{quote}

\end{fulllineitems}

\index{get\_lineloads\_for\_plotting() (beamon.database.Database method)@\spxentry{get\_lineloads\_for\_plotting()}\spxextra{beamon.database.Database method}}

\begin{fulllineitems}
\phantomsection\label{\detokenize{api:beamon.database.Database.get_lineloads_for_plotting}}\pysiglinewithargsret{\sphinxbfcode{\sphinxupquote{get\_lineloads\_for\_plotting}}}{}{}~\begin{description}
\item[{Gets lineloads index with starting and ending node coordinates}] \leavevmode
and lineload forces for plotting.

\end{description}
\begin{quote}\begin{description}
\item[{Returns}] \leavevmode
{[}link\_id, x1,y1,z1,x2,y2,z2, qx,qy,qz,qw{]}

\item[{Return type}] \leavevmode
float matrix

\end{description}\end{quote}

\end{fulllineitems}

\index{get\_link\_direction() (beamon.database.Database method)@\spxentry{get\_link\_direction()}\spxextra{beamon.database.Database method}}

\begin{fulllineitems}
\phantomsection\label{\detokenize{api:beamon.database.Database.get_link_direction}}\pysiglinewithargsret{\sphinxbfcode{\sphinxupquote{get\_link\_direction}}}{\emph{\DUrole{n}{index1}}, \emph{\DUrole{n}{index2}}}{}
Get the normalized direction of the Link specified by starting and ending node indices.
\begin{quote}\begin{description}
\item[{Parameters}] \leavevmode\begin{itemize}
\item {} 
\sphinxstyleliteralstrong{\sphinxupquote{index1}} \textendash{} index of the starting node

\item {} 
\sphinxstyleliteralstrong{\sphinxupquote{index2}} \textendash{} index of the ending node

\end{itemize}

\item[{Returns}] \leavevmode
{[}n\_x, n\_y, n\_z{]}

\end{description}\end{quote}

\end{fulllineitems}

\index{get\_link\_length() (beamon.database.Database method)@\spxentry{get\_link\_length()}\spxextra{beamon.database.Database method}}

\begin{fulllineitems}
\phantomsection\label{\detokenize{api:beamon.database.Database.get_link_length}}\pysiglinewithargsret{\sphinxbfcode{\sphinxupquote{get\_link\_length}}}{\emph{\DUrole{n}{index}}}{}
Gets the length of the link with the index.
\begin{quote}\begin{description}
\item[{Parameters}] \leavevmode
\sphinxstyleliteralstrong{\sphinxupquote{index}} \textendash{} of the link in the table Links

\item[{Returns}] \leavevmode
float

\end{description}\end{quote}

\end{fulllineitems}

\index{get\_link\_nodes() (beamon.database.Database method)@\spxentry{get\_link\_nodes()}\spxextra{beamon.database.Database method}}

\begin{fulllineitems}
\phantomsection\label{\detokenize{api:beamon.database.Database.get_link_nodes}}\pysiglinewithargsret{\sphinxbfcode{\sphinxupquote{get\_link\_nodes}}}{\emph{\DUrole{n}{link\_id}}}{}
Gets links starting and ending nodes indices.
\begin{quote}\begin{description}
\item[{Parameters}] \leavevmode
\sphinxstyleliteralstrong{\sphinxupquote{link\_id}} (\sphinxstyleliteralemphasis{\sphinxupquote{integer}}) \textendash{} link id

\item[{Returns}] \leavevmode
{[}n1\_id, n2\_id{]}

\end{description}\end{quote}

\end{fulllineitems}

\index{get\_link\_orientation() (beamon.database.Database method)@\spxentry{get\_link\_orientation()}\spxextra{beamon.database.Database method}}

\begin{fulllineitems}
\phantomsection\label{\detokenize{api:beamon.database.Database.get_link_orientation}}\pysiglinewithargsret{\sphinxbfcode{\sphinxupquote{get\_link\_orientation}}}{\emph{\DUrole{n}{index}}}{}
gets all local z\sphinxhyphen{}axes orientation information \sphinxhyphen{} plane normal vector (x axes) and v vector components
(orientation definition vector for z axes).
\begin{quote}\begin{description}
\item[{Parameters}] \leavevmode
\sphinxstyleliteralstrong{\sphinxupquote{index}} (\sphinxstyleliteralemphasis{\sphinxupquote{integer}}) \textendash{} link index number

\item[{Returns}] \leavevmode
x,y,z vectors for local system of the link

\end{description}\end{quote}

\end{fulllineitems}

\index{get\_link\_orientations() (beamon.database.Database method)@\spxentry{get\_link\_orientations()}\spxextra{beamon.database.Database method}}

\begin{fulllineitems}
\phantomsection\label{\detokenize{api:beamon.database.Database.get_link_orientations}}\pysiglinewithargsret{\sphinxbfcode{\sphinxupquote{get\_link\_orientations}}}{}{}
gets all local z\sphinxhyphen{}axes orientation information \sphinxhyphen{} plane normal vector (x axes) and v vector components
(orientation definition vector for z axes).
\begin{quote}\begin{description}
\item[{Returns}] \leavevmode
x,y,z vectors for local system of all links

\end{description}\end{quote}

\end{fulllineitems}

\index{get\_link\_with\_id() (beamon.database.Database method)@\spxentry{get\_link\_with\_id()}\spxextra{beamon.database.Database method}}

\begin{fulllineitems}
\phantomsection\label{\detokenize{api:beamon.database.Database.get_link_with_id}}\pysiglinewithargsret{\sphinxbfcode{\sphinxupquote{get\_link\_with\_id}}}{\emph{\DUrole{n}{link\_id}}}{}
Gets link data with a certain id
\begin{quote}\begin{description}
\item[{Parameters}] \leavevmode
\sphinxstyleliteralstrong{\sphinxupquote{link\_id}} (\sphinxstyleliteralemphasis{\sphinxupquote{integer}}) \textendash{} link id

\item[{Returns}] \leavevmode
link data

\end{description}\end{quote}

\end{fulllineitems}

\index{get\_links() (beamon.database.Database method)@\spxentry{get\_links()}\spxextra{beamon.database.Database method}}

\begin{fulllineitems}
\phantomsection\label{\detokenize{api:beamon.database.Database.get_links}}\pysiglinewithargsret{\sphinxbfcode{\sphinxupquote{get\_links}}}{}{}
Gets Links information for visualization puposes as line segment information.
\begin{quote}\begin{description}
\item[{Returns}] \leavevmode
numpy array for line segments with the form {[}x, y, z{]}

\end{description}\end{quote}

\end{fulllineitems}

\index{get\_links\_ending\_points() (beamon.database.Database method)@\spxentry{get\_links\_ending\_points()}\spxextra{beamon.database.Database method}}

\begin{fulllineitems}
\phantomsection\label{\detokenize{api:beamon.database.Database.get_links_ending_points}}\pysiglinewithargsret{\sphinxbfcode{\sphinxupquote{get\_links\_ending\_points}}}{}{}
gets x,y,z coordinates of all links ending points.
This could be used to plot local coordinate systems of the links.
\begin{quote}\begin{description}
\item[{Returns}] \leavevmode
{[}x,y,z{]}

\end{description}\end{quote}

\end{fulllineitems}

\index{get\_links\_indexes() (beamon.database.Database method)@\spxentry{get\_links\_indexes()}\spxextra{beamon.database.Database method}}

\begin{fulllineitems}
\phantomsection\label{\detokenize{api:beamon.database.Database.get_links_indexes}}\pysiglinewithargsret{\sphinxbfcode{\sphinxupquote{get\_links\_indexes}}}{}{}
Gets the column ‘id’ of the Links table in the database.
\begin{quote}\begin{description}
\item[{Returns}] \leavevmode
numpy array {[}id{]}

\end{description}\end{quote}

\end{fulllineitems}

\index{get\_links\_middle\_points() (beamon.database.Database method)@\spxentry{get\_links\_middle\_points()}\spxextra{beamon.database.Database method}}

\begin{fulllineitems}
\phantomsection\label{\detokenize{api:beamon.database.Database.get_links_middle_points}}\pysiglinewithargsret{\sphinxbfcode{\sphinxupquote{get\_links\_middle\_points}}}{}{}
gets x,y,z coordinates of all links middle points.
This could be used to plot local coordinate systems of the links
\begin{quote}\begin{description}
\item[{Returns}] \leavevmode
{[}x,y,z{]}

\end{description}\end{quote}

\end{fulllineitems}

\index{get\_links\_profiles() (beamon.database.Database method)@\spxentry{get\_links\_profiles()}\spxextra{beamon.database.Database method}}

\begin{fulllineitems}
\phantomsection\label{\detokenize{api:beamon.database.Database.get_links_profiles}}\pysiglinewithargsret{\sphinxbfcode{\sphinxupquote{get\_links\_profiles}}}{}{}
Gets profile data from links provided their profile id is set.
\begin{quote}\begin{description}
\item[{Returns}] \leavevmode
{[}E, G, A, Iy, Iz, Kv{]}

\end{description}\end{quote}

\end{fulllineitems}

\index{get\_links\_starting\_points() (beamon.database.Database method)@\spxentry{get\_links\_starting\_points()}\spxextra{beamon.database.Database method}}

\begin{fulllineitems}
\phantomsection\label{\detokenize{api:beamon.database.Database.get_links_starting_points}}\pysiglinewithargsret{\sphinxbfcode{\sphinxupquote{get\_links\_starting\_points}}}{}{}
gets x,y,z coordinates of all links start points.
This is being used to plot local coordinate systems of the links.
\begin{quote}\begin{description}
\item[{Returns}] \leavevmode
{[}x,y,z{]}

\end{description}\end{quote}

\end{fulllineitems}

\index{get\_loads() (beamon.database.Database method)@\spxentry{get\_loads()}\spxextra{beamon.database.Database method}}

\begin{fulllineitems}
\phantomsection\label{\detokenize{api:beamon.database.Database.get_loads}}\pysiglinewithargsret{\sphinxbfcode{\sphinxupquote{get\_loads}}}{}{}
Gets all the nodes forces from Load table.
\begin{quote}\begin{description}
\item[{Returns}] \leavevmode
list matrix

\end{description}\end{quote}

\end{fulllineitems}

\index{get\_loads\_for\_plotting() (beamon.database.Database method)@\spxentry{get\_loads\_for\_plotting()}\spxextra{beamon.database.Database method}}

\begin{fulllineitems}
\phantomsection\label{\detokenize{api:beamon.database.Database.get_loads_for_plotting}}\pysiglinewithargsret{\sphinxbfcode{\sphinxupquote{get\_loads\_for\_plotting}}}{}{}
Make a 9 column matrix containing (x,y,z,u,v,w, m\_x,m\_y,m\_z) which are the positions and components of the
forces applied to nodes.
This method is necessary for plotting the node loads accordingly.
\begin{quote}\begin{description}
\item[{Returns}] \leavevmode
numpy array with {[}x,y,z,u,v,w, m\_x,m\_y,m\_z{]}/ None: if empty loads table

\end{description}\end{quote}

\end{fulllineitems}

\index{get\_lost\_dependencies() (beamon.database.Database method)@\spxentry{get\_lost\_dependencies()}\spxextra{beamon.database.Database method}}

\begin{fulllineitems}
\phantomsection\label{\detokenize{api:beamon.database.Database.get_lost_dependencies}}\pysiglinewithargsret{\sphinxbfcode{\sphinxupquote{get\_lost\_dependencies}}}{}{}
This method gets nodes indexes from Links and node loads that are not connected anymore.
Warning: Using this method means you could have lost data and generated a delete anomaly in the database
TODO: use this method in integrity check

\end{fulllineitems}

\index{get\_node() (beamon.database.Database method)@\spxentry{get\_node()}\spxextra{beamon.database.Database method}}

\begin{fulllineitems}
\phantomsection\label{\detokenize{api:beamon.database.Database.get_node}}\pysiglinewithargsret{\sphinxbfcode{\sphinxupquote{get\_node}}}{\emph{\DUrole{n}{index}}}{}
gets the nodes x,y,z coordinates.
\begin{quote}\begin{description}
\item[{Parameters}] \leavevmode
\sphinxstyleliteralstrong{\sphinxupquote{index}} \textendash{} of the node

\item[{Returns}] \leavevmode
None if node not found / numpy array with {[}x,y,z{]}

\end{description}\end{quote}

\end{fulllineitems}

\index{get\_node\_id() (beamon.database.Database method)@\spxentry{get\_node\_id()}\spxextra{beamon.database.Database method}}

\begin{fulllineitems}
\phantomsection\label{\detokenize{api:beamon.database.Database.get_node_id}}\pysiglinewithargsret{\sphinxbfcode{\sphinxupquote{get\_node\_id}}}{\emph{\DUrole{n}{raw\_index}}}{}
gets the index of the node from row number.
\begin{quote}\begin{description}
\item[{Parameters}] \leavevmode
\sphinxstyleliteralstrong{\sphinxupquote{raw\_index}} \textendash{} the index of the raw

\item[{Returns}] \leavevmode
\sphinxhyphen{}1 if not found. Otherwise node id will be given

\end{description}\end{quote}

\end{fulllineitems}

\index{get\_node\_index() (beamon.database.Database method)@\spxentry{get\_node\_index()}\spxextra{beamon.database.Database method}}

\begin{fulllineitems}
\phantomsection\label{\detokenize{api:beamon.database.Database.get_node_index}}\pysiglinewithargsret{\sphinxbfcode{\sphinxupquote{get\_node\_index}}}{\emph{\DUrole{n}{x}}, \emph{\DUrole{n}{y}}, \emph{\DUrole{n}{z}}}{}
gets the index of the node from its given coordinates.
\begin{quote}\begin{description}
\item[{Parameters}] \leavevmode\begin{itemize}
\item {} 
\sphinxstyleliteralstrong{\sphinxupquote{x}} \textendash{} coordinate

\item {} 
\sphinxstyleliteralstrong{\sphinxupquote{y}} \textendash{} coordinate

\item {} 
\sphinxstyleliteralstrong{\sphinxupquote{z}} \textendash{} coordinate

\end{itemize}

\item[{Returns}] \leavevmode
\sphinxhyphen{}1 if not found. Otherwise node index will be given

\end{description}\end{quote}

\end{fulllineitems}

\index{get\_nodes() (beamon.database.Database method)@\spxentry{get\_nodes()}\spxextra{beamon.database.Database method}}

\begin{fulllineitems}
\phantomsection\label{\detokenize{api:beamon.database.Database.get_nodes}}\pysiglinewithargsret{\sphinxbfcode{\sphinxupquote{get\_nodes}}}{\emph{\DUrole{n}{type}\DUrole{o}{=}\DUrole{default_value}{None}}}{}
Gets all nodes coordinates with/without a certain type.
\begin{quote}\begin{description}
\item[{Returns}] \leavevmode
numpy array for each node with the form {[}x, y, z{]}

\end{description}\end{quote}

\end{fulllineitems}

\index{get\_nodes\_indexes\_with\_bc() (beamon.database.Database method)@\spxentry{get\_nodes\_indexes\_with\_bc()}\spxextra{beamon.database.Database method}}

\begin{fulllineitems}
\phantomsection\label{\detokenize{api:beamon.database.Database.get_nodes_indexes_with_bc}}\pysiglinewithargsret{\sphinxbfcode{\sphinxupquote{get\_nodes\_indexes\_with\_bc}}}{}{}
Gets all nodes id numbers which have boundary conditions defined on them. Similar to get\_all\_dof but delivers
nodes indices instead of all nodes information.
\begin{quote}\begin{description}
\item[{Returns}] \leavevmode
nodes indices

\item[{Return type}] \leavevmode
integer vector

\end{description}\end{quote}

\end{fulllineitems}

\index{get\_profiles() (beamon.database.Database method)@\spxentry{get\_profiles()}\spxextra{beamon.database.Database method}}

\begin{fulllineitems}
\phantomsection\label{\detokenize{api:beamon.database.Database.get_profiles}}\pysiglinewithargsret{\sphinxbfcode{\sphinxupquote{get\_profiles}}}{}{}
Gets all the profiles which are element properties {[}E G A I\_y I\_z K\_z{]}.
\begin{quote}\begin{description}
\item[{Returns}] \leavevmode
list

\end{description}\end{quote}

\end{fulllineitems}

\index{get\_results\_view\_settings() (beamon.database.Database method)@\spxentry{get\_results\_view\_settings()}\spxextra{beamon.database.Database method}}

\begin{fulllineitems}
\phantomsection\label{\detokenize{api:beamon.database.Database.get_results_view_settings}}\pysiglinewithargsret{\sphinxbfcode{\sphinxupquote{get\_results\_view\_settings}}}{}{}
Get all results view settings from table VisualizerSettings.
\begin{quote}\begin{description}
\item[{Returns}] \leavevmode
results view map

\item[{Return type}] \leavevmode
\sphinxhref{https://docs.python.org/3/library/stdtypes.html\#dict}{dict}

\end{description}\end{quote}

\end{fulllineitems}

\index{get\_view\_settings() (beamon.database.Database method)@\spxentry{get\_view\_settings()}\spxextra{beamon.database.Database method}}

\begin{fulllineitems}
\phantomsection\label{\detokenize{api:beamon.database.Database.get_view_settings}}\pysiglinewithargsret{\sphinxbfcode{\sphinxupquote{get\_view\_settings}}}{}{}
Get all view settings from table VisualizerSettings.
\begin{quote}\begin{description}
\item[{Returns}] \leavevmode
list of params / None if table is empty

\end{description}\end{quote}

\end{fulllineitems}

\index{has\_links() (beamon.database.Database method)@\spxentry{has\_links()}\spxextra{beamon.database.Database method}}

\begin{fulllineitems}
\phantomsection\label{\detokenize{api:beamon.database.Database.has_links}}\pysiglinewithargsret{\sphinxbfcode{\sphinxupquote{has\_links}}}{}{}
Checks if there is any links in Links table.
\begin{quote}\begin{description}
\item[{Returns}] \leavevmode
True/False

\end{description}\end{quote}

\end{fulllineitems}

\index{has\_nodes() (beamon.database.Database method)@\spxentry{has\_nodes()}\spxextra{beamon.database.Database method}}

\begin{fulllineitems}
\phantomsection\label{\detokenize{api:beamon.database.Database.has_nodes}}\pysiglinewithargsret{\sphinxbfcode{\sphinxupquote{has\_nodes}}}{}{}
Checks if there is any nodes in Nodes table.
\begin{quote}\begin{description}
\item[{Returns}] \leavevmode
True/False

\end{description}\end{quote}

\end{fulllineitems}

\index{import\_text() (beamon.database.Database method)@\spxentry{import\_text()}\spxextra{beamon.database.Database method}}

\begin{fulllineitems}
\phantomsection\label{\detokenize{api:beamon.database.Database.import_text}}\pysiglinewithargsret{\sphinxbfcode{\sphinxupquote{import\_text}}}{\emph{\DUrole{n}{path}}}{}
This method is for importing geometry files to the database according to specified syntax in documentation.
Keywords: {\color{red}\bfseries{}*}node, {\color{red}\bfseries{}*}element, {\color{red}\bfseries{}*}profile, {\color{red}\bfseries{}*}load, {\color{red}\bfseries{}*}lineload.
Tables will be checked for integrity after import.
\begin{quote}\begin{description}
\item[{Parameters}] \leavevmode
\sphinxstyleliteralstrong{\sphinxupquote{path}} \textendash{} of the text file to be imported

\end{description}\end{quote}

\end{fulllineitems}

\index{is\_settings\_preset() (beamon.database.Database method)@\spxentry{is\_settings\_preset()}\spxextra{beamon.database.Database method}}

\begin{fulllineitems}
\phantomsection\label{\detokenize{api:beamon.database.Database.is_settings_preset}}\pysiglinewithargsret{\sphinxbfcode{\sphinxupquote{is\_settings\_preset}}}{}{}
Checks if settings table has already been changed or not.
\begin{quote}\begin{description}
\item[{Returns}] \leavevmode
True / False

\end{description}\end{quote}

\end{fulllineitems}

\index{make\_dummy\_file() (beamon.database.Database method)@\spxentry{make\_dummy\_file()}\spxextra{beamon.database.Database method}}

\begin{fulllineitems}
\phantomsection\label{\detokenize{api:beamon.database.Database.make_dummy_file}}\pysiglinewithargsret{\sphinxbfcode{\sphinxupquote{make\_dummy\_file}}}{\emph{\DUrole{n}{ram}}, \emph{\DUrole{n}{name}}}{}
Create a new sqlite database according to given settings. This method contains CREATE queries for database
tables.
\begin{quote}\begin{description}
\item[{Parameters}] \leavevmode\begin{itemize}
\item {} 
\sphinxstyleliteralstrong{\sphinxupquote{name}} \textendash{} name of the database file. Default name is a string containing a timestamp as follows:
Database\_dd\sphinxhyphen{}mm\sphinxhyphen{}yyyy\_hh\sphinxhyphen{}mm\sphinxhyphen{}ss

\item {} 
\sphinxstyleliteralstrong{\sphinxupquote{ram}} \textendash{} default:False. if True the database wil be loaded into the ram and not saved to hard drive

\end{itemize}

\end{description}\end{quote}

\end{fulllineitems}

\index{make\_edof() (beamon.database.Database method)@\spxentry{make\_edof()}\spxextra{beamon.database.Database method}}

\begin{fulllineitems}
\phantomsection\label{\detokenize{api:beamon.database.Database.make_edof}}\pysiglinewithargsret{\sphinxbfcode{\sphinxupquote{make\_edof}}}{}{}
Create element degrees of freedom numbers by iterating over nodes.
This method sorts all dof numbers according to connectivity information
This method is used when reading input files
TODO: consider Joints
:return:

\end{fulllineitems}

\index{remove\_lineload() (beamon.database.Database method)@\spxentry{remove\_lineload()}\spxextra{beamon.database.Database method}}

\begin{fulllineitems}
\phantomsection\label{\detokenize{api:beamon.database.Database.remove_lineload}}\pysiglinewithargsret{\sphinxbfcode{\sphinxupquote{remove\_lineload}}}{\emph{\DUrole{n}{index}}}{}
Removes a certain load with index from lineload table.
\begin{quote}\begin{description}
\item[{Parameters}] \leavevmode
\sphinxstyleliteralstrong{\sphinxupquote{index}} (\sphinxstyleliteralemphasis{\sphinxupquote{integer}}) \textendash{} lineload index number

\end{description}\end{quote}

\end{fulllineitems}

\index{remove\_link() (beamon.database.Database method)@\spxentry{remove\_link()}\spxextra{beamon.database.Database method}}

\begin{fulllineitems}
\phantomsection\label{\detokenize{api:beamon.database.Database.remove_link}}\pysiglinewithargsret{\sphinxbfcode{\sphinxupquote{remove\_link}}}{\emph{\DUrole{n}{index}}}{}
Drops a link from the links table.
\begin{quote}\begin{description}
\item[{Parameters}] \leavevmode
\sphinxstyleliteralstrong{\sphinxupquote{index}} \textendash{} index of the link

\item[{Returns}] \leavevmode
True: if link has been successfully droppend, False: otherwise

\end{description}\end{quote}

\end{fulllineitems}

\index{remove\_link\_with\_node() (beamon.database.Database method)@\spxentry{remove\_link\_with\_node()}\spxextra{beamon.database.Database method}}

\begin{fulllineitems}
\phantomsection\label{\detokenize{api:beamon.database.Database.remove_link_with_node}}\pysiglinewithargsret{\sphinxbfcode{\sphinxupquote{remove\_link\_with\_node}}}{\emph{\DUrole{n}{index}}}{}
Drops a link from the Links table in relation with the node with index.

Warning: Link should definitely have the node with index
:param index: of the node

\end{fulllineitems}

\index{remove\_load() (beamon.database.Database method)@\spxentry{remove\_load()}\spxextra{beamon.database.Database method}}

\begin{fulllineitems}
\phantomsection\label{\detokenize{api:beamon.database.Database.remove_load}}\pysiglinewithargsret{\sphinxbfcode{\sphinxupquote{remove\_load}}}{\emph{\DUrole{n}{index}}}{}
Remove load with certain index number.
\begin{quote}\begin{description}
\item[{Parameters}] \leavevmode
\sphinxstyleliteralstrong{\sphinxupquote{index}} (\sphinxstyleliteralemphasis{\sphinxupquote{integer}}) \textendash{} load index.

\end{description}\end{quote}

\end{fulllineitems}

\index{remove\_loads\_with\_node() (beamon.database.Database method)@\spxentry{remove\_loads\_with\_node()}\spxextra{beamon.database.Database method}}

\begin{fulllineitems}
\phantomsection\label{\detokenize{api:beamon.database.Database.remove_loads_with_node}}\pysiglinewithargsret{\sphinxbfcode{\sphinxupquote{remove\_loads\_with\_node}}}{\emph{\DUrole{n}{index}}}{}
remove all forces and moments in relation with the node.
\begin{quote}\begin{description}
\item[{Parameters}] \leavevmode
\sphinxstyleliteralstrong{\sphinxupquote{index}} \textendash{} of the node

\item[{Returns}] \leavevmode
True

\end{description}\end{quote}

\end{fulllineitems}

\index{remove\_node() (beamon.database.Database method)@\spxentry{remove\_node()}\spxextra{beamon.database.Database method}}

\begin{fulllineitems}
\phantomsection\label{\detokenize{api:beamon.database.Database.remove_node}}\pysiglinewithargsret{\sphinxbfcode{\sphinxupquote{remove\_node}}}{\emph{\DUrole{n}{index}}}{}
Drops a node from the nodes table.
\sphinxstylestrong{Warning}: Dependent links and loads entries will be deleted.
\begin{quote}\begin{description}
\item[{Parameters}] \leavevmode
\sphinxstyleliteralstrong{\sphinxupquote{index}} \textendash{} index of the node

\item[{Returns}] \leavevmode
True: if node has been successfully dropped, False: otherwise

\end{description}\end{quote}

\end{fulllineitems}

\index{remove\_profile() (beamon.database.Database method)@\spxentry{remove\_profile()}\spxextra{beamon.database.Database method}}

\begin{fulllineitems}
\phantomsection\label{\detokenize{api:beamon.database.Database.remove_profile}}\pysiglinewithargsret{\sphinxbfcode{\sphinxupquote{remove\_profile}}}{\emph{\DUrole{n}{index}}}{}
Removes the profile with index
\begin{quote}\begin{description}
\item[{Parameters}] \leavevmode
\sphinxstyleliteralstrong{\sphinxupquote{index}} \textendash{} of the profile

\item[{Returns}] \leavevmode
True: successful / False: not successful

\end{description}\end{quote}

\end{fulllineitems}

\index{set\_grid\_settings() (beamon.database.Database method)@\spxentry{set\_grid\_settings()}\spxextra{beamon.database.Database method}}

\begin{fulllineitems}
\phantomsection\label{\detokenize{api:beamon.database.Database.set_grid_settings}}\pysiglinewithargsret{\sphinxbfcode{\sphinxupquote{set\_grid\_settings}}}{\emph{\DUrole{n}{params}}}{}
Sets all grid settings to the table VisualizerSettings.
\begin{quote}\begin{description}
\item[{Parameters}] \leavevmode
\sphinxstyleliteralstrong{\sphinxupquote{params}} (\sphinxstyleliteralemphasis{\sphinxupquote{doubles list}}) \textendash{} (grid\_x1origin,grid\_x2origin, grid\_x3origin, grid\_n1, grid\_n2, grid\_n3, grid\_xtick, grid\_ytick)

\end{description}\end{quote}

\end{fulllineitems}

\index{set\_results\_view\_settings() (beamon.database.Database method)@\spxentry{set\_results\_view\_settings()}\spxextra{beamon.database.Database method}}

\begin{fulllineitems}
\phantomsection\label{\detokenize{api:beamon.database.Database.set_results_view_settings}}\pysiglinewithargsret{\sphinxbfcode{\sphinxupquote{set\_results\_view\_settings}}}{\emph{\DUrole{n}{map}}}{}
Sets all results view settings to the table VisualizerSettings
\begin{quote}\begin{description}
\item[{Parameters}] \leavevmode
\sphinxstyleliteralstrong{\sphinxupquote{map}} (\sphinxhref{https://docs.python.org/3/library/stdtypes.html\#dict}{\sphinxstyleliteralemphasis{\sphinxupquote{dict}}}) \textendash{} dictionary to map each parameter

\end{description}\end{quote}

\end{fulllineitems}

\index{set\_view\_settings() (beamon.database.Database method)@\spxentry{set\_view\_settings()}\spxextra{beamon.database.Database method}}

\begin{fulllineitems}
\phantomsection\label{\detokenize{api:beamon.database.Database.set_view_settings}}\pysiglinewithargsret{\sphinxbfcode{\sphinxupquote{set\_view\_settings}}}{\emph{\DUrole{n}{params}}}{}
Sets all view settings to the table VisualizerSettings.
\begin{quote}\begin{description}
\item[{Parameters}] \leavevmode
\sphinxstyleliteralstrong{\sphinxupquote{params}} (\sphinxstyleliteralemphasis{\sphinxupquote{doubles list}}) \textendash{} list with 1 numbers following order (view\_dim)

\end{description}\end{quote}

\end{fulllineitems}

\index{sort\_edof() (beamon.database.Database method)@\spxentry{sort\_edof()}\spxextra{beamon.database.Database method}}

\begin{fulllineitems}
\phantomsection\label{\detokenize{api:beamon.database.Database.sort_edof}}\pysiglinewithargsret{\sphinxbfcode{\sphinxupquote{sort\_edof}}}{}{}
Resets edof numbers counting to 1 while keeping previous connection information.

\end{fulllineitems}

\index{sort\_lineload\_indexes() (beamon.database.Database method)@\spxentry{sort\_lineload\_indexes()}\spxextra{beamon.database.Database method}}

\begin{fulllineitems}
\phantomsection\label{\detokenize{api:beamon.database.Database.sort_lineload_indexes}}\pysiglinewithargsret{\sphinxbfcode{\sphinxupquote{sort\_lineload\_indexes}}}{}{}
Reset id column in Lineload table.
All Lineloads primary keys will be resorted to ascending ordered integers from 1 to n (n is number of Lineloads).

\end{fulllineitems}

\index{sort\_link\_indexes() (beamon.database.Database method)@\spxentry{sort\_link\_indexes()}\spxextra{beamon.database.Database method}}

\begin{fulllineitems}
\phantomsection\label{\detokenize{api:beamon.database.Database.sort_link_indexes}}\pysiglinewithargsret{\sphinxbfcode{\sphinxupquote{sort\_link\_indexes}}}{}{}
Reset id column in Links table.
All Links primary keys will be resorted to ascending ordered integers from 1 to n (n is number of links).
Warning: This method also resets foreign keys in Lineload table that reference Links

\end{fulllineitems}

\index{sort\_load\_indexes() (beamon.database.Database method)@\spxentry{sort\_load\_indexes()}\spxextra{beamon.database.Database method}}

\begin{fulllineitems}
\phantomsection\label{\detokenize{api:beamon.database.Database.sort_load_indexes}}\pysiglinewithargsret{\sphinxbfcode{\sphinxupquote{sort\_load\_indexes}}}{}{}
Reset id column in Load table.
All Load primary keys will be resorted to ascending ordered integers from 1 to n (n is number of Load).

\end{fulllineitems}

\index{sort\_node\_indexes() (beamon.database.Database method)@\spxentry{sort\_node\_indexes()}\spxextra{beamon.database.Database method}}

\begin{fulllineitems}
\phantomsection\label{\detokenize{api:beamon.database.Database.sort_node_indexes}}\pysiglinewithargsret{\sphinxbfcode{\sphinxupquote{sort\_node\_indexes}}}{}{}
Reset id column in Nodes table.
All nodes primary keys will be resorted to ascending ordered integers from 1 to n (n is number of nodes).
Warning: This method also resets foreign keys in Links and Loads tables that reference Nodes.
\begin{quote}\begin{description}
\item[{Returns}] \leavevmode
True: success/False: did not reset anything

\end{description}\end{quote}

\end{fulllineitems}

\index{sort\_profile\_indexes() (beamon.database.Database method)@\spxentry{sort\_profile\_indexes()}\spxextra{beamon.database.Database method}}

\begin{fulllineitems}
\phantomsection\label{\detokenize{api:beamon.database.Database.sort_profile_indexes}}\pysiglinewithargsret{\sphinxbfcode{\sphinxupquote{sort\_profile\_indexes}}}{}{}
Reset id column in Profile table.
All profiles primary keys will be resorted to ascending ordered integers from 1 to n (n is number of nodes).
Warning: This method also resets foreign keys in Links table that reference Profile

\end{fulllineitems}


\end{fulllineitems}



\subsection{Indices and tables}
\label{\detokenize{api:indices-and-tables}}\begin{itemize}
\item {} 
\DUrole{xref,std,std-ref}{genindex}

\item {} 
\DUrole{xref,std,std-ref}{modindex}

\item {} 
\DUrole{xref,std,std-ref}{search}

\end{itemize}


\section{Contributing}
\label{\detokenize{contrib:contributing}}\label{\detokenize{contrib::doc}}

\subsection{Programming Environment}
\label{\detokenize{env:programming-environment}}\label{\detokenize{env::doc}}
Making your environment optimized to deal with large projects such as Beamon is necessary to ensure low\sphinxhyphen{}cost
maintenance.


\subsubsection{Python Environment}
\label{\detokenize{env:python-environment}}
Windows users should install the Anaconda environment.

\begin{sphinxadmonition}{note}{Note:}
All packages should be installed via pip and not from conda.
\end{sphinxadmonition}

Some anaconda packages are not up to date with the latest releases and will cause compatibility problems.
Nevertheless using Anaconda has proven to be most efficient.

Please Install Anaconda from the \sphinxhref{https://www.anaconda.com/}{official website}


\subsubsection{IDE}
\label{\detokenize{env:ide}}
The most convenient IDE for projects like Beamon is PyCharm, especially PyCharm Professional.

Please Install PyCharm from the \sphinxhref{https://www.jetbrains.com/pycharm/}{official website}


\subsubsection{Project Requirements}
\label{\detokenize{env:project-requirements}}
“Package requirements” are the \sphinxstylestrong{absolute minimum} of python packages that are required to perform \sphinxstylestrong{all}
operation in the project.

All required packages are in \sphinxstyleemphasis{requirements.txt} file and can be installed via pip using the command:

\begin{sphinxVerbatim}[commandchars=\\\{\}]
\PYG{n}{pip} \PYG{n}{install} \PYG{o}{\PYGZhy{}}\PYG{n}{r} \PYG{n}{requirements}\PYG{o}{.}\PYG{n}{txt}
\end{sphinxVerbatim}

or using the command:

\begin{sphinxVerbatim}[commandchars=\\\{\}]
\PYG{n}{make} \PYG{n}{install\PYGZus{}requirements}
\end{sphinxVerbatim}

on your environment console.

For windows users, \sphinxstylestrong{pywin32}, which will be used to build Windows executables, will be required in addition
to \sphinxstyleemphasis{requirements.txt} packages.

\begin{sphinxadmonition}{warning}{Warning:}
Package requirements should be updated regularly to guarantee that latest package releases work.
\end{sphinxadmonition}


\subsection{Project Structure}
\label{\detokenize{structure:project-structure}}\label{\detokenize{structure::doc}}
For structuring python projects read the Hitchhiker’s Guid to Python \sphinxhref{https://docs.python-guide.org/writing/structure/}{here}.


\subsubsection{Directives Structure}
\label{\detokenize{structure:directives-structure}}
\begin{sphinxVerbatim}[commandchars=\\\{\}]
\PYG{g+gp}{\PYGZgt{}\PYGZgt{}\PYGZgt{} }\PYG{n}{tree} \PYG{o}{/}\PYG{n}{F}
\PYG{g+go}{.}
\PYG{g+go}{|   .gitattributes}
\PYG{g+go}{│   .gitignore}
\PYG{g+go}{│   LICENSE}
\PYG{g+go}{│   make.bat}
\PYG{g+go}{│   make.sh}
\PYG{g+go}{│   README.md}
\PYG{g+go}{│   requirements.txt}
\PYG{g+go}{│   runbeamon.spec}
\PYG{g+go}{│   setup.py}
\PYG{g+go}{├───beamon}
\PYG{g+go}{│   ├───resources}
\PYG{g+go}{│   │   ├───custom\PYGZus{}icons}
\PYG{g+go}{│   │   └───ui}
\PYG{g+go}{│   ├───ui}
\PYG{g+go}{│   │   ├───expansions}
\PYG{g+go}{│   │   ├───settings}
\PYG{g+go}{│   │   ├───tables}
\PYG{g+go}{├───build}
\PYG{g+go}{├───dist}
\PYG{g+go}{├───docs}
\PYG{g+go}{│   ├───build}
\PYG{g+go}{│   ├───release}
\PYG{g+go}{│   └───source}
\PYG{g+go}{│       ├───resources}
\PYG{g+go}{│       │   └───figures}
\PYG{g+go}{│       └───\PYGZus{}static}
\PYG{g+go}{├───releases}
\PYG{g+go}{│   └───alpha}
\PYG{g+go}{├───resources}
\PYG{g+go}{└───tests}
\end{sphinxVerbatim}


\subsubsection{Discription}
\label{\detokenize{structure:discription}}

\paragraph{The Actual Module}
\label{\detokenize{structure:the-actual-module}}
This is where the core focus of the repository is.


\begin{savenotes}\sphinxattablestart
\centering
\begin{tabulary}{\linewidth}[t]{|T|T|}
\hline

Location
&
./beamon/
\\
\hline
Purpose
&
Programs code
\\
\hline
\end{tabulary}
\par
\sphinxattableend\end{savenotes}

When user lunches Beamon as a python package using python interpreter, e.g.

\begin{sphinxVerbatim}[commandchars=\\\{\}]
\PYG{n}{python} \PYG{o}{\PYGZhy{}}\PYG{n}{m} \PYG{n}{beamon}
\end{sphinxVerbatim}

file \_\_main\_\_.py will be executed.
On the other hand \_\_main.py will be executed if beamon executable e.g. \sphinxtitleref{runbeamon.exe} is called using bash


\paragraph{Release Files}
\label{\detokenize{structure:release-files}}

\begin{savenotes}\sphinxattablestart
\centering
\begin{tabulary}{\linewidth}[t]{|T|T|}
\hline

Location
&
./releases
\\
\hline
Purpose
&
All executables releases
\\
\hline
\end{tabulary}
\par
\sphinxattableend\end{savenotes}


\begin{savenotes}\sphinxattablestart
\centering
\begin{tabulary}{\linewidth}[t]{|T|T|}
\hline

Location
&
./docs/release
\\
\hline
Purpose
&
latest documentation release
\\
\hline
\end{tabulary}
\par
\sphinxattableend\end{savenotes}


\paragraph{Resources}
\label{\detokenize{structure:resources}}

\begin{savenotes}\sphinxattablestart
\centering
\begin{tabulary}{\linewidth}[t]{|T|T|}
\hline

Location
&
./beamon/resources
\\
\hline
Purpose
&
program dependent resources for running
\\
\hline
\end{tabulary}
\par
\sphinxattableend\end{savenotes}


\begin{savenotes}\sphinxattablestart
\centering
\begin{tabulary}{\linewidth}[t]{|T|T|}
\hline

Location
&
./resources
\\
\hline
Purpose
&
project dependent resources
\\
\hline
\end{tabulary}
\par
\sphinxattableend\end{savenotes}


\paragraph{License}
\label{\detokenize{structure:license}}

\begin{savenotes}\sphinxattablestart
\centering
\begin{tabulary}{\linewidth}[t]{|T|T|}
\hline

Location
&
./LICENSE
\\
\hline
Purpose
&
Copyright
\\
\hline
\end{tabulary}
\par
\sphinxattableend\end{savenotes}


\paragraph{Setup.py}
\label{\detokenize{structure:setup-py}}

\begin{savenotes}\sphinxattablestart
\centering
\begin{tabulary}{\linewidth}[t]{|T|T|}
\hline

Location
&
./setup.py
\\
\hline
Purpose
&
Package and distribution management.
\\
\hline
\end{tabulary}
\par
\sphinxattableend\end{savenotes}


\paragraph{Make}
\label{\detokenize{structure:make}}

\begin{savenotes}\sphinxattablestart
\centering
\begin{tabulary}{\linewidth}[t]{|T|T|}
\hline

Location
&
./make.bat and ./make.sh
\\
\hline
Purpose
&
make for both linux and windows
\\
\hline
\end{tabulary}
\par
\sphinxattableend\end{savenotes}


\paragraph{PyInstaller Specification:}
\label{\detokenize{structure:pyinstaller-specification}}
This file is auto generated from make and excluded from remote repository. It can be modified for testing purposes.


\begin{savenotes}\sphinxattablestart
\centering
\begin{tabulary}{\linewidth}[t]{|T|T|}
\hline

Location
&
./runbeamon.spec
\\
\hline
Purpose
&
secure distribution management.
\\
\hline
\end{tabulary}
\par
\sphinxattableend\end{savenotes}


\paragraph{Distribution files}
\label{\detokenize{structure:distribution-files}}
This directory is also autogenerated and excluded from remote repository.


\begin{savenotes}\sphinxattablestart
\centering
\begin{tabulary}{\linewidth}[t]{|T|T|}
\hline

Location
&
./dist/
\\
\hline
Purpose
&
distributing Beamon as package or executable
\\
\hline
\end{tabulary}
\par
\sphinxattableend\end{savenotes}


\paragraph{Requirements File}
\label{\detokenize{structure:requirements-file}}
Contains all dependencies for the project.


\begin{savenotes}\sphinxattablestart
\centering
\begin{tabulary}{\linewidth}[t]{|T|T|}
\hline

Location
&
./requirements.txt
\\
\hline
Purpose
&
Developers evironment dependencies
\\
\hline
\end{tabulary}
\par
\sphinxattableend\end{savenotes}


\paragraph{Documentation}
\label{\detokenize{structure:documentation}}
It is worth noting that the \sphinxstylestrong{release} directory inside \sphinxstylestrong{docs} is the latest documentation release.


\begin{savenotes}\sphinxattablestart
\centering
\begin{tabulary}{\linewidth}[t]{|T|T|}
\hline

Location
&
./docs/
\\
\hline
Purpose
&
Package reference documentation.
\\
\hline
\end{tabulary}
\par
\sphinxattableend\end{savenotes}


\paragraph{Test Suite}
\label{\detokenize{structure:test-suite}}

\begin{savenotes}\sphinxattablestart
\centering
\begin{tabulary}{\linewidth}[t]{|T|T|}
\hline

Location
&
./tests/
\\
\hline
Purpose
&
Package integration and unit tests.
\\
\hline
\end{tabulary}
\par
\sphinxattableend\end{savenotes}


\subsection{Implementation Details}
\label{\detokenize{implementation:implementation-details}}\label{\detokenize{implementation::doc}}
This chapter include implementation details from database structure to UML class diagrams.
It should help develop an understanding of the project for newcomers.


\subsubsection{The Database}
\label{\detokenize{implementation:the-database}}
The Database is implemented using SQLite3 in Python. Using SQLite database files as project files is quite convenient.
Therefore we will be referring to database files with project files.


\paragraph{Database Structure (alpha release)}
\label{\detokenize{implementation:database-structure-alpha-release}}
The following diagram shows an entity\sphinxhyphen{}relationship diagram (ER\sphinxhyphen{}Diagram) of the alpha release.

\begin{figure}[htbp]
\centering
\capstart

\noindent\sphinxincludegraphics{{er_diagram_alpha}.png}
\caption{ER\sphinxhyphen{}Diagram of the alpha release.}\label{\detokenize{implementation:id1}}\end{figure}

Some entities have composite attributes. For example, Links has attribute \(\vec{v}\), which practically
have to be divided up into v\_x, v\_y, and v\_z.
\sphinxstyleemphasis{dof} represents which degree of freedom is locked/unlocked (0/1) on a node to save boundary conditions.
\sphinxstyleemphasis{edof} is element degree of freedom which consists of 12 integer numbers.


\paragraph{SQLite Queries}
\label{\detokenize{implementation:sqlite-queries}}
The following sql query shows how to pass parameters to an sql query.
cur is connection object cursor.

\begin{sphinxVerbatim}[commandchars=\\\{\}]
\PYG{k+kn}{import} \PYG{n+nn}{sqlite3}
\PYG{n}{conn} \PYG{o}{=} \PYG{n}{sqlite3}\PYG{o}{.}\PYG{n}{connect}\PYG{p}{(}\PYG{l+s+s2}{\PYGZdq{}}\PYG{l+s+s2}{path\PYGZus{}to\PYGZus{}database}\PYG{l+s+s2}{\PYGZdq{}}\PYG{p}{)}
\PYG{n}{cur} \PYG{o}{=} \PYG{n}{conn}\PYG{o}{.}\PYG{n}{cursor}\PYG{p}{(}\PYG{p}{)}
\PYG{n}{cur}\PYG{o}{.}\PYG{n}{execute}\PYG{p}{(}\PYG{l+s+s1}{\PYGZsq{}}\PYG{l+s+s1}{SELECT id FROM Nodes WHERE x=}\PYG{l+s+si}{\PYGZob{}x\PYGZcb{}}\PYG{l+s+s1}{ and y=}\PYG{l+s+si}{\PYGZob{}y\PYGZcb{}}\PYG{l+s+s1}{ and z=}\PYG{l+s+si}{\PYGZob{}z\PYGZcb{}}\PYG{l+s+s1}{;}\PYG{l+s+s1}{\PYGZsq{}}\PYG{o}{.}\PYG{n}{format}\PYG{p}{(}\PYG{n}{x}\PYG{o}{=}\PYG{l+m+mi}{0}\PYG{p}{,} \PYG{n}{y}\PYG{o}{=}\PYG{l+m+mf}{0.1}\PYG{p}{,} \PYG{n}{z}\PYG{o}{=}\PYG{o}{\PYGZhy{}}\PYG{l+m+mi}{1}\PYG{p}{)}\PYG{p}{)}
\end{sphinxVerbatim}

Another way of passing parameters to sql queries is using the tuple params.

\begin{sphinxVerbatim}[commandchars=\\\{\}]
\PYG{n}{conn} \PYG{o}{=} \PYG{n}{sqlite3}\PYG{o}{.}\PYG{n}{connect}\PYG{p}{(}\PYG{l+s+s2}{\PYGZdq{}}\PYG{l+s+s2}{path\PYGZus{}to\PYGZus{}database}\PYG{l+s+s2}{\PYGZdq{}}\PYG{p}{)}
\PYG{n}{cur} \PYG{o}{=} \PYG{n}{conn}\PYG{o}{.}\PYG{n}{cursor}\PYG{p}{(}\PYG{p}{)}
\PYG{n}{params} \PYG{o}{=} \PYG{p}{(}\PYG{l+m+mi}{0}\PYG{p}{,} \PYG{l+m+mf}{0.1}\PYG{p}{,} \PYG{o}{\PYGZhy{}}\PYG{l+m+mi}{1}\PYG{p}{)}
\PYG{n}{cur}\PYG{o}{.}\PYG{n}{execute}\PYG{p}{(}\PYG{l+s+s1}{\PYGZsq{}}\PYG{l+s+s1}{SELECT id FROM Nodes WHERE x=}\PYG{l+s+s1}{\PYGZob{}}\PYG{l+s+s1}{?\PYGZcb{} and y=}\PYG{l+s+s1}{\PYGZob{}}\PYG{l+s+s1}{?\PYGZcb{} and z=}\PYG{l+s+s1}{\PYGZob{}}\PYG{l+s+s1}{?\PYGZcb{};}\PYG{l+s+s1}{\PYGZsq{}}\PYG{p}{,} \PYG{n}{params}\PYG{p}{)}
\end{sphinxVerbatim}

You can also obtain panda Dataframes using sql queries. It is very convenient especially when using QTableViews to show
data tables.

\begin{sphinxVerbatim}[commandchars=\\\{\}]
\PYG{n}{conn} \PYG{o}{=} \PYG{n}{sqlite3}\PYG{o}{.}\PYG{n}{connect}\PYG{p}{(}\PYG{l+s+s2}{\PYGZdq{}}\PYG{l+s+s2}{path\PYGZus{}to\PYGZus{}database}\PYG{l+s+s2}{\PYGZdq{}}\PYG{p}{)}
\PYG{k+kn}{import} \PYG{n+nn}{pandas}\PYG{n+nn}{.}\PYG{n+nn}{io}\PYG{n+nn}{.}\PYG{n+nn}{sql} \PYG{k}{as} \PYG{n+nn}{sql}
\PYG{n}{sql}\PYG{o}{.}\PYG{n}{read\PYGZus{}sql}\PYG{p}{(}\PYG{l+s+s2}{\PYGZdq{}\PYGZdq{}\PYGZdq{}}\PYG{l+s+s2}{SELECT E, G, A, Iy, Iz, kv, k from Profile}\PYG{l+s+s2}{\PYGZdq{}\PYGZdq{}\PYGZdq{}}\PYG{p}{,} \PYG{n}{conn}\PYG{p}{)}
\end{sphinxVerbatim}


\paragraph{Reading Geometry Files Efficiently}
\label{\detokenize{implementation:reading-geometry-files-efficiently}}
Geometry files are, for modifiability purposes, text files with ASCII characters.
Reading large text files into SQLite database could cause run\sphinxhyphen{}time problems.
\sphinxstyleemphasis{import\_text} method in \sphinxstyleemphasis{Database} class avoids those problems by firstly reading the text file in a buffered CSV file,
which will be transformed into pandas dataframe, which is imported eventually into the database efficiently.

The following activity diagram summarizes the steps taken in \sphinxstyleemphasis{import\_text} to import geometry files.

\begin{figure}[htbp]
\centering
\capstart

\noindent\sphinxincludegraphics{{activity_import_text}.png}
\caption{UML activity diagram showing steps to import geometry files in acitvity import\_text.}\label{\detokenize{implementation:id2}}\end{figure}


\paragraph{Resetting Primary Keys}
\label{\detokenize{implementation:resetting-primary-keys}}
After deleting rows in SQLite3 tables, primary keys would have gaps. To prevent that from happening, primary keys have to be altered.
Resetting primary keys involves changing foreign keys as well.

The following methods reset primary keys for each table:
\begin{itemize}
\item {} 
\sphinxstyleemphasis{sort\_node\_indexes}

\item {} 
\sphinxstyleemphasis{sort\_profile\_indexes}

\item {} 
\sphinxstyleemphasis{sort\_link\_indexes}

\item {} 
\sphinxstyleemphasis{sort\_load\_indexes}

\item {} 
\sphinxstyleemphasis{sort\_lineload\_indexes}

\end{itemize}


\subsubsection{PySide2 or PyQt5}
\label{\detokenize{implementation:pyside2-or-pyqt5}}
PySid2 and PyQt5 have almost the same api.
But it is worth mentioning that using PyQt5 is more beneficial in the long run.
PySide development lagged behind PyQt and PySide2 supports only Linux and MacOs Platforms.
On the Other hand PySide have very convenient Libraries for 3D animations that PyQt5 dont have.

Eventually PyQt5 was chosen due to major popularity and community support.


\subsubsection{Package vs. Executable}
\label{\detokenize{implementation:package-vs-executable}}
Running Beamon as a package is not much challenging as an Executable that has been securely compressed using
e.g. PyInstaller (among other).

In some cases you need to differentiate between the two situations.
Some dependencies have to be formatted first before they get used. For example all .ui files will be formatted
using pyuic5 from xml to python at every run in the main script. This will prevent mistakes like forgetting
to update the user interface that is being used after modification.
In the Executable version of the program this is no longer needed. The program will be in a frozen state and there
are no updates to any files anymore.

For this reason there are two main scripts for each of the program versions
\sphinxstylestrong{\_\_main\_\_.py} for Beamon as a package and \sphinxstylestrong{\_\_main.py} for Beamon as an executable.

\sphinxstylestrong{\_\_main.py} has no formatting commands whereas \sphinxstylestrong{\_\_main\_\_.py} uses pyuic5 module to format local files like .ui and
.rst


\subsubsection{Class Structure}
\label{\detokenize{implementation:class-structure}}
The following class diagram shows the whole class structure in Beamon project.

\begin{figure}[htbp]
\centering
\capstart

\noindent\sphinxincludegraphics{{class_complex}.jpg}
\caption{UML class diagram showing Beamon project structure.}\label{\detokenize{implementation:id3}}\end{figure}

There are two main scripts \sphinxstyleemphasis{\_\_main\_\_} and \sphinxstyleemphasis{\_\_main} that instantiate the application.
\sphinxstyleemphasis{\_\_main is for executables and does not contain formatting instructions. Both main scripts are connected to the
*Main} GUI class.

\sphinxstyleemphasis{Main} resides in the \sphinxstyleemphasis{ui} package, which contains everything related to the user interface.
As you can see \sphinxstyleemphasis{Main} inherits from \sphinxstyleemphasis{QMainWindow}, which is a single tone PyQt5 class.

\sphinxstyleemphasis{VisualizerMainForm} class is responsible for visualization. It contains \sphinxstyleemphasis{PyVistaWidget} which is the visualization
component based on PyVista and should therefore inherit from \sphinxstyleemphasis{QtInteractor}.
Furthermore, the visualizer can call helper dialogs like \sphinxstyleemphasis{GridSettingsMainForm} to control grid settings.

\sphinxstyleemphasis{BeamSizeMainForm} class can be called from \sphinxstyleemphasis{Main}. BeamSize is an expansion of the basic functionality of Beamon and
should therefore reside in the \sphinxstyleemphasis{expansions} package. \sphinxstyleemphasis{BeamSizeMainform} contains a PyQtGraph plot widget. \sphinxstyleemphasis{PlotWidget}
should therefore inherit from \sphinxstyleemphasis{QGraphicsView}.

\sphinxstyleemphasis{QDockWidget} can be triggered from \sphinxstyleemphasis{Main} and contains a widget that could be dynamically changed. \sphinxstyleemphasis{QDockWidget}
shows either \sphinxstyleemphasis{IntroMainForm} or \sphinxstyleemphasis{TableMainForm}.

\sphinxstyleemphasis{IntroMainForm} contains a web browser that shows this documentation in PDF\sphinxhyphen{}format.

Inside \sphinxstyleemphasis{tables} package you will find classes that follow the model\sphinxhyphen{}view\sphinxhyphen{}controller design pattern. \sphinxstyleemphasis{TableMainForm}
contains the view, which is QTableView. \sphinxstyleemphasis{InLineEditDelegate} is the controller. All models like \sphinxstyleemphasis{NodeTableModel} inherit
from \sphinxstyleemphasis{QAbstractTableModel}.


\subsubsection{The Simulation}
\label{\detokenize{implementation:the-simulation}}
The following diagram shows how the simulation in Beamon is implemented. Three parts are involved in the Simulation,
\sphinxstyleemphasis{Database}, \sphinxstyleemphasis{Core} and \sphinxstyleemphasis{CALFEM}. Each part has it’s own responsibility. \sphinxstyleemphasis{Database} is responsible for obtaining and
saving data. \sphinxstyleemphasis{Core} is responsible for the calculations implemented in Beamon. \sphinxstyleemphasis{CALFEM} is the Calfem\sphinxhyphen{}Python Package
which is responsible for the most complex calculation.

\begin{figure}[htbp]
\centering
\capstart

\noindent\sphinxincludegraphics{{activity_simulation}.jpg}
\caption{UML activity diagram showing static simulation steps.}\label{\detokenize{implementation:id4}}\end{figure}


\subsection{How TOs}
\label{\detokenize{howto:how-tos}}\label{\detokenize{howto::doc}}
This chapter introduces some best practices and tutorials on how to implement some necessary features in the project.


\subsubsection{How To Build This Documentation}
\label{\detokenize{howto:how-to-build-this-documentation}}
Steps to build sphinx documentation:
\begin{enumerate}
\sphinxsetlistlabels{\arabic}{enumi}{enumii}{}{.}%
\item {} 
Install perl

\item {} 
Install MikTex and latexmk package in MikTex

\item {} 
Install sphinx package in your Python environment

\end{enumerate}

On your environment command line run following command from project root directory:

\begin{sphinxVerbatim}[commandchars=\\\{\}]
\PYG{n}{python} \PYG{o}{\PYGZhy{}}\PYG{n}{m} \PYG{n}{sphinx} \PYG{o}{\PYGZhy{}}\PYG{n}{b} \PYG{n}{latex} \PYG{n}{docs}\PYGZbs{}\PYG{n}{source} \PYG{n}{docs}\PYGZbs{}\PYG{n}{build}\PYGZbs{}\PYG{n}{latex}
\end{sphinxVerbatim}

to build the latex version of the documentation. And then run

\begin{sphinxVerbatim}[commandchars=\\\{\}]
\PYG{n}{cd} \PYG{n}{docs}\PYGZbs{}\PYG{n}{build}\PYGZbs{}\PYG{n}{latex} \PYG{o}{\PYGZam{}}\PYG{o}{\PYGZam{}} \PYG{n}{make}
\end{sphinxVerbatim}

to compile the latex documentation.

Or run:

\begin{sphinxVerbatim}[commandchars=\\\{\}]
\PYG{n}{python} \PYG{o}{\PYGZhy{}}\PYG{n}{m} \PYG{n}{sphinx} \PYG{o}{\PYGZhy{}}\PYG{n}{b} \PYG{n}{html} \PYG{n}{docs}\PYGZbs{}\PYG{n}{source} \PYG{n}{docs}\PYGZbs{}\PYG{n}{build}\PYGZbs{}\PYG{n}{html}
\end{sphinxVerbatim}

to build the html version of the documentation.


\subsubsection{How to Style QTableView Cells}
\label{\detokenize{howto:how-to-style-qtableview-cells}}
\sphinxurl{https://www.learnpyqt.com/tutorials/qtableview-modelviews-numpy-pandas/}


\renewcommand{\indexname}{Python Module Index}
\begin{sphinxtheindex}
\let\bigletter\sphinxstyleindexlettergroup
\bigletter{b}
\item\relax\sphinxstyleindexentry{beamon.core}\sphinxstyleindexpageref{api:\detokenize{module-beamon.core}}
\item\relax\sphinxstyleindexentry{beamon.database}\sphinxstyleindexpageref{api:\detokenize{module-beamon.database}}
\item\relax\sphinxstyleindexentry{beamon.simulation}\sphinxstyleindexpageref{api:\detokenize{module-beamon.simulation}}
\end{sphinxtheindex}

\renewcommand{\indexname}{Index}
\printindex
\end{document}